% Options for packages loaded elsewhere
\PassOptionsToPackage{unicode}{hyperref}
\PassOptionsToPackage{hyphens}{url}
%
\documentclass[
]{article}
\usepackage{amsmath,amssymb}
\usepackage{lmodern}
\usepackage{ifxetex,ifluatex}
\ifnum 0\ifxetex 1\fi\ifluatex 1\fi=0 % if pdftex
  \usepackage[T1]{fontenc}
  \usepackage[utf8]{inputenc}
  \usepackage{textcomp} % provide euro and other symbols
\else % if luatex or xetex
  \usepackage{unicode-math}
  \defaultfontfeatures{Scale=MatchLowercase}
  \defaultfontfeatures[\rmfamily]{Ligatures=TeX,Scale=1}
  \setmainfont[BoldFont = SF Pro Text Medium]{SF Pro Text Light}
  \setmathfont[]{Fira Math}
\fi
% Use upquote if available, for straight quotes in verbatim environments
\IfFileExists{upquote.sty}{\usepackage{upquote}}{}
\IfFileExists{microtype.sty}{% use microtype if available
  \usepackage[]{microtype}
  \UseMicrotypeSet[protrusion]{basicmath} % disable protrusion for tt fonts
}{}
\makeatletter
\@ifundefined{KOMAClassName}{% if non-KOMA class
  \IfFileExists{parskip.sty}{%
    \usepackage{parskip}
  }{% else
    \setlength{\parindent}{0pt}
    \setlength{\parskip}{6pt plus 2pt minus 1pt}}
}{% if KOMA class
  \KOMAoptions{parskip=half}}
\makeatother
\usepackage{xcolor}
\IfFileExists{xurl.sty}{\usepackage{xurl}}{} % add URL line breaks if available
\IfFileExists{bookmark.sty}{\usepackage{bookmark}}{\usepackage{hyperref}}
\hypersetup{
  pdftitle={PHIL 640: Week Two - Moral Arguments},
  pdfauthor={Brian Weatherson},
  hidelinks,
  pdfcreator={LaTeX via pandoc}}
\urlstyle{same} % disable monospaced font for URLs
\usepackage[margin=1.5in]{geometry}
\usepackage{graphicx}
\makeatletter
\def\maxwidth{\ifdim\Gin@nat@width>\linewidth\linewidth\else\Gin@nat@width\fi}
\def\maxheight{\ifdim\Gin@nat@height>\textheight\textheight\else\Gin@nat@height\fi}
\makeatother
% Scale images if necessary, so that they will not overflow the page
% margins by default, and it is still possible to overwrite the defaults
% using explicit options in \includegraphics[width, height, ...]{}
\setkeys{Gin}{width=\maxwidth,height=\maxheight,keepaspectratio}
% Set default figure placement to htbp
\makeatletter
\def\fps@figure{htbp}
\makeatother
\setlength{\emergencystretch}{3em} % prevent overfull lines
\providecommand{\tightlist}{%
  \setlength{\itemsep}{0pt}\setlength{\parskip}{0pt}}
\setcounter{secnumdepth}{-\maxdimen} % remove section numbering
\linespread{1.18}
\ifluatex
  \usepackage{selnolig}  % disable illegal ligatures
\fi
\newlength{\cslhangindent}
\setlength{\cslhangindent}{1.5em}
\newlength{\csllabelwidth}
\setlength{\csllabelwidth}{3em}
\newenvironment{CSLReferences}[2] % #1 hanging-ident, #2 entry spacing
 {% don't indent paragraphs
  \setlength{\parindent}{0pt}
  % turn on hanging indent if param 1 is 1
  \ifodd #1 \everypar{\setlength{\hangindent}{\cslhangindent}}\ignorespaces\fi
  % set entry spacing
  \ifnum #2 > 0
  \setlength{\parskip}{#2\baselineskip}
  \fi
 }%
 {}
\usepackage{calc}
\newcommand{\CSLBlock}[1]{#1\hfill\break}
\newcommand{\CSLLeftMargin}[1]{\parbox[t]{\csllabelwidth}{#1}}
\newcommand{\CSLRightInline}[1]{\parbox[t]{\linewidth - \csllabelwidth}{#1}\break}
\newcommand{\CSLIndent}[1]{\hspace{\cslhangindent}#1}

\title{PHIL 640: Week Two - Moral Arguments}
\author{Brian Weatherson}
\date{September 13, 2021}

\begin{document}
\maketitle

\hypertarget{the-is-ought-gap}{%
\subsection{The Is-Ought Gap}\label{the-is-ought-gap}}

There is this famous principle, usually credited to Hume, which says
that moral and factual claims occupy distinct realms. The bumper sticker
version of it - and it actually wouldn't be a bad bumper sticker - says
\textbf{No ought from an is}. More precisely, there is no way to validly
deduce a moral claim from a factual one.

Now there are plenty of cases where this seems like a good bit of
advice. For instance, it tells us immediately that there is something
wrong with the following bit of reasoning.

\begin{enumerate}
\def\labelenumi{\arabic{enumi}.}
\tightlist
\item
  Public opinion tolerates 100 American deaths per day in car crashes.
\item
  So it should tolerate 100 deaths per day from other causes.
\end{enumerate}

And from 2 you might get some claim like vaccine mandates are a bad
thing or something. But we should already get off the boat before that.
This looks like a bad ought-from-an-is argument. Someone who approves of
this way of looking at arguments will say that the argument has a hidden
premise. It should have gone like this.

\begin{enumerate}
\def\labelenumi{\arabic{enumi}.}
\tightlist
\item
  Public opinion tolerates 100 American deaths per day in car crashes.
\item
  It should tolerate anything similar to what it actually tolerates.
\item
  So it should tolerate 100 deaths per day from other causes.
\end{enumerate}

And making premise 2 visible is useful, because it is kind of obviously
false. After all, premise 2 entails that we should tolerate everything
we do tolerate, and that's pretty clearly false.

The lesson here is not meant to be something particular about vaccines
or public opinion or anything. It's rather a certain way of looking at
moral argumentation that says that there is always a moral premise in
arguments for moral conclusions. Now the question before us is whether
this `always' is true. You could think that the heuristic \emph{Look for
hidden moral premises} is a good heuristic, and still think that it
isn't always true that there are always these moral premises. So let's
turn to that question.

In ``The Autonomy of Ethics,'' Arthur Prior (1960) has these influential
counterexamples to the idea that an argument with a moral conclusion
always has moral premises. Let's work through the three kinds of cases
he gives in turn.

The first is just or-introduction, though Prior's example should come
with some kind of content warning.\footnote{Calling it a trigger warning
  seems gauche in this context. Note that Prior was at this stage a New
  Zealander living in England.}

\begin{enumerate}
\def\labelenumi{\arabic{enumi}.}
\tightlist
\item
  Tea-drinking is common in England.
\item
  So, either tea-drinking is common in England, or all New Zealanders
  should be shot.
\end{enumerate}

Prior notes that if you think the conclusion is non-ethical, then the
following works as a counterexample.

\begin{enumerate}
\def\labelenumi{\arabic{enumi}.}
\tightlist
\item
  Either tea-drinking is common in England, or all New Zealanders should
  be shot.
\item
  Tea-drinking is not common in England.
\item
  So, all New Zealanders should be shot.
\end{enumerate}

The big assumption here, and one we'll come back to, is that a statement
is either factual or descriptive. If you want to say of this statement
that it is sort of one, sort of the other, then you can (sort of) avoid
the objection. But that's really all you need.

The second argument involves null quantification.

\begin{enumerate}
\def\labelenumi{\arabic{enumi}.}
\tightlist
\item
  No one is over 20 foot tall.
\item
  So, no one over 20 foot tall may sit in an ordinary chair.
\end{enumerate}

Again, if you want to deny that the conclusion is an evaluative
statement, we could offer the following argument.

\begin{enumerate}
\def\labelenumi{\arabic{enumi}.}
\tightlist
\item
  No one over 20 foot tall may sit in an ordinary chair.
\item
  Brian is over 20 foot tall.
\item
  So Brian may not sit in an ordinary chair.
\end{enumerate}

The third example is in some ways the closest to our interests in this
course. Prior's own version is I think a bit needlessly complicated, so
let's substitute a slightly simpler one. (I've seen people quote
something like this version when describing Prior; this simplification
isn't at all original.)

\begin{enumerate}
\def\labelenumi{\arabic{enumi}.}
\tightlist
\item
  Brian is a baker.
\item
  So Brian ought to do everything that all bakers ought to do.
\end{enumerate}

Note that the conclusion is not just a restatement of the premise. It
could be that the premise is false and the conclusion true. Imagine that
the only thing anyone ought to do is worship God. Then the conclusion
would be true even if (as is fact the case) I am not a baker. But the
conclusion does follow from the premise.

The big question here is whether the conclusion is really a moral
conclusion. Prior notes one big reason that you might think that it is
not a moral conclusion; the term `ought' can be substituted with all
sorts of other terms and the validity of the argument is preserved.

\begin{enumerate}
\def\labelenumi{\arabic{enumi}.}
\tightlist
\item
  Brian is a baker.
\item
  So Brian wants to do everything that all bakers want to do.
\end{enumerate}

\hypertarget{a-semantic-response-to-prior}{%
\subsection{A Semantic Response to
Prior}\label{a-semantic-response-to-prior}}

Next I want to go over a recent response to Prior by Gillian Russell.
This is set out in part in a paper she co-authored with Greg Restall
(Restall and Russell 2010), and partially in a sole authored paper
(Russell 2011). I'm not 100\% sure I agree with Russell's response, but
I think it's a very good way of setting out how Prior's arguments look
from a certain perspective. And it's that perspective that I want us to
be able to focus on for the next little while.

There are two big assumptions behind Prior's paper.

\begin{description}
\tightlist
\item[Syntactic Distinction]
The distinction between `is' sentences and `ought' sentences is broadly
syntactic; it's about which words appear in the sentence, and where they
appear in them.
\item[Exhaustive Distinction]
The distinction between `is' sentences and `ought' sentences is
exhaustive; every sentence is in one or the other.
\end{description}

Russell's response rejects both of these assumptions. She wants to
understand the distinction semantically, not syntactically, and she
rejects the exhaustiveness assumption. The second of those rejections is
easy to explain - she thinks some sentences are neither descriptive nor
evaluation. And in particular, disjunctions like \emph{Tea-drinking is
common in England or all New Zealanders should be shot} is on neither
side of the divide. But the first rejection is more interesting for our
purposes, and takes a bit longer to explain.

To offer a semantic conception of something, in this sense, is to
explain it in terms of models, and in particular using semantic concepts
- like truth, satisfaction, reference etc - in models. Here we'll be
focussing on truth in a model. What's distinctive here are the kind of
models being used. The models are going to be sets of worlds, as in
standard models. But the worlds themselves will have structure. They
will be ordered pairs \(\langle d, m \rangle\), where \(d\) is a full
description of the descriptive, `is,' part of the world, and \(m\) is
the moral theory in the world. Sentences will be true or false relative
to these pairs, not relative to purely descriptive claims. So if \(d\)
makes it true that Brian is a baker, and \(m\) makes it true that all
bakers should shave, then \(\langle d, m \rangle\) makes it true that
Brian should shave.

Now we could at this point spend a lot of time going into what kind of
thing \(m\) is, and what kind of options there are for it. But that
would be a different kind of course. Just note for now that \(m\) will
be some kind of mathematical structure whose ordinary interpretation
will be in terms of maximally thin moral/ethical/evaluative notions.

With this in mind, we can offer an account of what it is for a sentence
to be descriptive or evaluative.

\begin{itemize}
\tightlist
\item
  A sentence \(S\) is descriptive iff for any \(d, m, m^\prime\), the
  sentence has the same truth value in \(\langle d, m \rangle\) as it
  has in \(\langle d, m^\prime \rangle\).
\item
  A sentence \(S\) is evaluative iff for any \(d, d^\prime, m\), the
  sentence has the same truth value in \(\langle d, m \rangle\) as it
  has in \(\langle d^\prime, m \rangle\).
\end{itemize}

That is, descriptive sentences don't change their truth value when you
change the moral part of the world. And evaluative sentences don't
change their truth value when you change the descriptive part of the
world. And those both seem plausible enough claims

Now note on this picture that a lot of sentences will be neither
descriptive nor evaluative. A disjunction of a descriptive and an
evaluative sentence will be neither. A conjunction of a descriptive and
an evaluative sentence will be neither. Depending on how you understand
\(m\), all three of the conclusions in Prior's arguments will be
neither.

But that said, it will definitely be true that a descriptive sentence,
or set of descriptive sentences, never entails an evaluative sentence,
unless it's a trivial evaluative sentence like \emph{Killing is wrong or
it isn't}, or a contradictory descriptive sentence like \emph{Killing is
common and it isn't}. Why? Well, the conclusion is non-trivial, so it is
false somewhere. Call that somewhere \(\langle d_1, m_1 \rangle\). And
the premise is non-trivial, so it is true somewhere. Call that somewhere
\(\langle d_2, m_2 \rangle\). Since the premise is descriptive, it
doesn't change truth value when you change \(m\), so it is still true at
\(\langle d_2, m_1 \rangle\). And the conclusion is evaluative so it
doesn't change truth value when you change \(d\), so it is false at
\(d_2, m_1\). So there is a world, namely \(\langle d_2, m_1 \rangle\),
where the premise is true and the conclusion false. So the argument
isn't valid.

Is this enough to rescue the no ought from an is principle? I think to
answer that you need to answer two questions. First, how worried are you
that so many sentences turn out to be neither descriptive nor
evaluative? I'm a bit worried about this, but I'm not sure how worried I
should be. Second, how happy are you with understanding worlds as pairs
like \(\langle d, m \rangle\)? And this I think is the big question.

Here's what I take Russell to have made a decent case for. If you're
happy with understanding worlds as pairs like \(\langle d, m \rangle\)
then there is a response to objections like Prior's. It's not a
pain-free response - it involves having this huge class of
neither-descriptive-nor-evaluative sentences - but it's a response. And
honestly if you did like no ought from an is, you probably should have
been thinking of worlds as being something like these kinds of pairs all
along.

Now one way to think about the core puzzle we're going to be thinking
about for the next few weeks is whether the best theory of thick
concepts/terms is consistent with this way of thinking about worlds. One
way to think about Foot's very different challenge to no ought from an
is takes it to be a direct attack on that very way of thinking about
things. So let's turn to Foot's paper.

\hypertarget{anti-realism}{%
\subsection{Anti-Realism}\label{anti-realism}}

Let's start by looking at how Foot sees the debate about anti-realism (I
won't here distinguish expressivism from non-cognitivism unless I have
to) circa 1958. There are two things that I want to stress that may seem
different to how they seem nowadays.

One is that Foot does not agree with the relatively simple causal story
I told last week. I think that story is widely believed nowadays, but
since Foot was closer to the action than contemporary commentators, I
want to put her version on the table as well.

Here is the causal story I told. Moore proved that the only viable form
of realism was non-naturalism. Later philosophers agreed with the proof,
but couldn't handle the non-naturalism, so they gave up realism. That's
very much not the story Foot tells.

On Foot's telling, Moore's argument is little more than Hume's \emph{No
ought from an is} principle. And that principle needs support, it can't
be taken as given. And the support it gets comes from varieties of
anti-realism. It's because we think \emph{Murder is wrong} is an
imperative not an assertion that we think Moore's argument goes through,
and not vice versa. Now as I said, I'm not really sure that's the way
things played in the 1920s and 1930s, but I think Foot's view is worth
taking seriously.

The other thing that is striking is what she takes to be the main
problem with anti-realist theories: they can't explain moral arguments.
Now while this isn't a million miles from the Frege-Geach problem that
became the biggest issue, it is a real challenge.

It's clearly true that we do make moral arguments in our everyday life.
If the conclusions of these are means to be expressions or imperatives,
we need a story about how such arguments might work. The best
non-cognitivist theory has two parts. One is that moral arguments are
like practical arguments. Just like we might discuss what to do tonight
and conclude \emph{Let's go to a movie}, we can discuss morality and
conclude with an imperative. The other is that for these arguments to
work, there has to be something practical in the premises. Moral
arguments can't start from nothing; they have to start from some level
of moral agreement. It's this last condition that Foot disagrees with.
She thinks that arguments about morality don't need an agreement on
moral starting point any more than arguments about the shape of the
earth.

It turns out not to be crucial to the plot line of this paper, but it is
striking that at the top of page 505 Foot thinks that one of the
problems with anti-realism is that it would lead to subjectivism about
moral evidence. And this is bad, because subjectivism about evidence is
bad. What's striking about this from a contemporary perspective is that
subjectivism about evidence is taken much more seriously now than Foot
allows. It is very common to say that a person can decide for themselves
``what is evidence for monetary inflation or a tumour on the brain.'' If
evidence is probability raising, and probability is fairly permissive,
then this kind of subjectivism will be true. (We could go into a long
digression here about the particular puzzles of finding evidence
concerning inflation, but that's for a very different course.)

There are some amusing asides about anti-realism later in the paper. You
can think of much of the paper as a kind of dilemma for expressivists.
Either they are cognitivists (moral attitudes are beliefs) or
non-cognitivists (moral attitudes are not beliefs, and more like
feelings). The first horn gets you into problems like at the bottom of
508 - some things that are intuitively possible become impossible. But
if you're a non-cognitivist, you get into the troubles described on 509.
And that's where we get this amusing pair of lines.

\begin{quote}
To suggest that he could refuse to admit that certain behaviour was rude
because the right psychological state had not been induced, is as odd as
to suppose that one might refuse to speak of the world as round because
in spite of the good evidence of roundness a feeling of confidence in
the proposition had not been produced. When given good evidence it is
one's business to act on it, not to hang around waiting for the right
state of mind.
\end{quote}

There is also an interesting point at the top of 506 about the ``special
use of fact'' that the expressivist needs. Early expressivists, notably
Ayer, got into all sorts of trouble with triples like these.

\begin{enumerate}
\def\labelenumi{\arabic{enumi}.}
\tightlist
\item
  Murder is wrong.
\item
  There are no moral facts.
\item
  It is a fact that S, iff, S.
\end{enumerate}

By 1 and 3 we get that it is a fact that murder is wrong, seeming to
contradict the claim in 2 that there are no moral facts. The
expressivist needs a way out of this. The way they normally take is to
posit an ambiguity in `fact.' In the ordinary sense of `fact,' 2 is
false and 3 is true. In the special philosophers' sense, 2 is true and 3
is false. But this leads to two problems. One is that this special
philosophers' sense might involve just the kind of metaphysics that the
expressivists pointedly wanted to get rid of. The other, which Foot
rightly stresses, is that we haven't got a good account of this special
sense.

I think Foot here is getting at something that has become known as the
problem of creeping minimalism. Minimalism about truth and facts, like
in 3, is very attractive for expressivists. But it threatens to
undermine the very statability of their view. They need to say things
like 2 in order to be distinctive. But things like 3 make that hard to
say.

The standard solution, which again Foot sees coming, is to give an
explicitly contrastive account of this special philosophers' sense of
fact. (I think Allan Gibbard might be the clearest proponent of this
approach.) What we really mean by 2 is that things like \emph{Murder is
wrong} are just not made true in the same way that things like \emph{The
cat is on the mat} is made true. There are a lot of things we might say
about this contrastive approach, but let's just note one that's
particularly relevant to Foot. If you go this way, you can't use
expressivism/non-cognitivism to motivate a fact/value distinction,
because you need the fact/value distinction to even be able to state the
anti-realist view.

\hypertarget{rudeness}{%
\subsection{Rudeness}\label{rudeness}}

So the part of this paper that is most relevant to us, and really the
heart of the paper, is the discussion of rudeness. So let's spend a bit
of time going over it. And I want to separate out a few different things
Foot says about rudeness.

\hypertarget{rudeness-is-evaluative}{%
\subsubsection{Rudeness is Evaluative}\label{rudeness-is-evaluative}}

As Foot says on 507, she takes ``rude'' to be an evaluative word. To
call someone, or something, rude is already to make a negative
evaluation. It might not be an all-things-considered negative
evaluation, but it is a negative evaluation nonetheless.

This matters because it affects what we think a violation of the Humean
gap looks like. For Foot, the counterexamples will have this form.

\begin{enumerate}
\def\labelenumi{\arabic{enumi}.}
\tightlist
\item
  Fred did X.
\item
  So, Fred was rude.
\end{enumerate}

And that's it - a descriptive premise and an evaluative conclusion. This
isn't the only way one could use `rude' to get to a puzzle. You could
use it as the middle step, like in\ldots{}

\begin{enumerate}
\def\labelenumi{\arabic{enumi}.}
\tightlist
\item
  Fred did X.
\item
  So, Fred was rude.
\item
  So, Fred acted wrongly.
\end{enumerate}

And now you could argue that the middle step is either descriptive, in
which case the argument from 2 to 3 is a counterexample, or evaluative,
in which case the argument from 1 to 2 is a counterexample. That's not
what Foot does. She just uses the argument from 1 to 2.

And I suspect she does this because she doesn't think the argument from
2 to 3 is valid. As she notes, there are cases where being rude is the
right thing to do. She says she is setting those cases aside, though I
didn't at all see why this was playing fair. You can't just set aside
whatever you choose I think. But perhaps she is being more sensitive to
this point than that way of putting makes it seem.

Anyway, the thing I want to take away from this is that it is misleading
to treat her as holding that ``rude'' has descriptive and evaluative
elements. Rather, it's an evaluative term that has some descriptive
constraints.

\hypertarget{causing-offence}{%
\subsubsection{Causing Offence}\label{causing-offence}}

I couldn't always tell just what Foot thought rudeness was. The initial
account, on the bottom of 507, is

\begin{quote}
The right account of the situation in which it is correct to say that a
piece of behaviour is rude, is, I think, that this kind of behaviour
causes offence by indicating lack of respect.
\end{quote}

There are a few things to unpack there. For one, it isn't obvious that
this is a morally neutral description. The `lack of respect' condition
at the end is arguably evaluative. But maybe whether something
`indicates a lack of respect' is a descriptive condition even if
`respect' is itself a moral notion. We could also I think ask if
`offence' is evaluative or descriptive, but perhaps it is descriptive
enough to let through.

In this account, what matters is that the \textbf{kind} of behaviour
causes offence. On the next page, the inference is from behaviour
actually causing offence to it being rude. This seems odd twice over.
For one thing, the reference to lack of respect dropped out. For
another, we've gone from talk about kinds to talk about individual
actions.

An account in terms of kinds seems clearly superior. I can rudely gossip
about someone behind their back, but if I'm a skilled enough gossip I
will never cause offence because they will never find out it was me. Or
I can be rude to someone's face, but if they are distracted it may not
actually cause offence. Now what's true is that the kind of behaviour I
engage in on both occasions does cause offence.

Usually this talk about properties of kinds of activities does not mean
that a kind has a property as long as one instance has that property. A
single goat that tap-dances does not make goats the kind of thing that
tap-dance. So it might be that some action causes offence (by indicating
lack of respect) without being the kind of action that causes offence.
It might just have been that someone around took offence at very strange
things.

This raises a problem I think for the argument on 508. We have to be
very careful about how we specify what goes into O. The wording on 508
suggests it is that some offence was actually caused. But that is
neither necessary nor sufficient for rudeness. We can totally say that
offence was caused but the action wasn't rude; it's just that the other
person was hyper-sensitive. (A celebration of sporting success might
cause offence to very thin-skinned opponents without being rude.) I
think Foot can get out of the problem here by saying that O isn't a
claim that this action actually causes offence, but that it is a kind
that typically does. But this is a tricky point, and if we're going to
argue that there is a metaphysical necessitation here, we have to watch
out for slippages.

Relatedly, the last line of the paragraph that runs from 508 to 509
makes very little sense to me. It seems fine to apologise for rude
behaviour while acknowledging no offence was caused. In the middle of an
office rushing to meet a deadline, I might do something rude, quickly
apologise, even while noting that everyone is so busy that no one had
time to even be offended. There isn't anything incoherent about this at
all. So I don't really understand what the argument here is supposed to
be.

\hypertarget{no-room-for-rudeness}{%
\subsection{No Room for Rudeness}\label{no-room-for-rudeness}}

It's only a couple of sentences, but I want to end the discussion of
Foot with what she says in the middle of 509. This is the question that
will become centrally important in what we discuss in a couple of weeks
time. What should we say about the person who rejects the concept of
rudeness?

It's clear what Foot wants to say about the kind of person she has in
mind. (Which may not be the kind of person we have in mind.) They will
just not use terms like `rude,' and we can't have any kind of
conversation with them about rudeness, etiquette, or anything similar.

I worry here that there are two kinds of people that are relevant, and
Foot is running them together.

\begin{enumerate}
\def\labelenumi{\arabic{enumi}.}
\tightlist
\item
  People who think `rude' is like `phlogiston,' a term from a failed
  theory.
\item
  People who can reliably track rude behaviour, but do not believe that
  it has any moral significance.
\end{enumerate}

What Foot says makes sense about people in category 1. But I'm not sure
what she has to say about people in category 2. Indeed, it is arguable
that she doesn't really believe in category 2, she thinks this is sort
of a way of being in category 1.

I'll leave it there with the notes, though we've got plenty more to talk
about in class. And if we have time we'll get onto the short
introduction to thick concepts at the very end of Chapter 7 of
\emph{Ethics and the Limits of Philosophy}.

But we're not moving straight on to Williams. Instead we'll stick with
the chronological order, and next week look at John McDowell's
shapelessness thesis. This is interesting both for its connection to a
bit of 20th century philosophy that is talked about way less than just a
few years ago - Wittgenstein on rule following - but also because it
firmly connects thick concepts to moral realism.

\hypertarget{refs}{}
\begin{CSLReferences}{1}{0}
\leavevmode\hypertarget{ref-Prior1960}{}%
Prior, Arthur. 1960. {``The Autonomy of Ethics.''} \emph{Australasian
Journal of Philosophy} 38 (3): 199--206.

\leavevmode\hypertarget{ref-RestallRussell2010}{}%
Restall, Greg, and Gillian Russell. 2010. {``Barriers to Implication.''}
In \emph{Hume on Is and Ought}, edited by Charles Pigden, 243--59.
London: Palgrave Macmillan.

\leavevmode\hypertarget{ref-Russell2011}{}%
Russell, Gillian. 2011. {``Indexicals, Context-Sensitivity and the
Failure of Implication.''} \emph{Synthese} 182 (2): 143--60.

\end{CSLReferences}

\end{document}
