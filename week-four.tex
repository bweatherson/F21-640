% Options for packages loaded elsewhere
\PassOptionsToPackage{unicode}{hyperref}
\PassOptionsToPackage{hyphens}{url}
%
\documentclass[
]{article}
\usepackage{amsmath,amssymb}
\usepackage{lmodern}
\usepackage{ifxetex,ifluatex}
\ifnum 0\ifxetex 1\fi\ifluatex 1\fi=0 % if pdftex
  \usepackage[T1]{fontenc}
  \usepackage[utf8]{inputenc}
  \usepackage{textcomp} % provide euro and other symbols
\else % if luatex or xetex
  \usepackage{unicode-math}
  \defaultfontfeatures{Scale=MatchLowercase}
  \defaultfontfeatures[\rmfamily]{Ligatures=TeX,Scale=1}
  \setmainfont[]{Baskerville}
\fi
% Use upquote if available, for straight quotes in verbatim environments
\IfFileExists{upquote.sty}{\usepackage{upquote}}{}
\IfFileExists{microtype.sty}{% use microtype if available
  \usepackage[]{microtype}
  \UseMicrotypeSet[protrusion]{basicmath} % disable protrusion for tt fonts
}{}
\makeatletter
\@ifundefined{KOMAClassName}{% if non-KOMA class
  \IfFileExists{parskip.sty}{%
    \usepackage{parskip}
  }{% else
    \setlength{\parindent}{0pt}
    \setlength{\parskip}{6pt plus 2pt minus 1pt}}
}{% if KOMA class
  \KOMAoptions{parskip=half}}
\makeatother
\usepackage{xcolor}
\IfFileExists{xurl.sty}{\usepackage{xurl}}{} % add URL line breaks if available
\IfFileExists{bookmark.sty}{\usepackage{bookmark}}{\usepackage{hyperref}}
\hypersetup{
  pdftitle={PHIL 640: Supporting Notes for Four Arguments},
  pdfauthor={Brian Weatherson},
  hidelinks,
  pdfcreator={LaTeX via pandoc}}
\urlstyle{same} % disable monospaced font for URLs
\usepackage[margin=1.4in]{geometry}
\usepackage{graphicx}
\makeatletter
\def\maxwidth{\ifdim\Gin@nat@width>\linewidth\linewidth\else\Gin@nat@width\fi}
\def\maxheight{\ifdim\Gin@nat@height>\textheight\textheight\else\Gin@nat@height\fi}
\makeatother
% Scale images if necessary, so that they will not overflow the page
% margins by default, and it is still possible to overwrite the defaults
% using explicit options in \includegraphics[width, height, ...]{}
\setkeys{Gin}{width=\maxwidth,height=\maxheight,keepaspectratio}
% Set default figure placement to htbp
\makeatletter
\def\fps@figure{htbp}
\makeatother
\setlength{\emergencystretch}{3em} % prevent overfull lines
\providecommand{\tightlist}{%
  \setlength{\itemsep}{0pt}\setlength{\parskip}{0pt}}
\setcounter{secnumdepth}{-\maxdimen} % remove section numbering
\linespread{1.17}
\ifluatex
  \usepackage{selnolig}  % disable illegal ligatures
\fi

\title{PHIL 640: Supporting Notes for Four Arguments}
\author{Brian Weatherson}
\date{September 20, 2021}

\begin{document}
\maketitle

\hypertarget{relativism}{%
\subsection{Relativism}\label{relativism}}

The orthodox approach to semantics involves a commitment to something
like the correspondance theory of truth. Declarative sentences are
(typically) true or false, and when they are true, this is because they
correspond to how things are. That last sentence needs a lot of spelling
out. How can a sentence correspond to anything, apart from another
sentence? Well, a sentence has a \textbf{content}. And that content can
be true or false relative to various things. What things? Well, those
are the matter of how things are in the platitude.

But how are things? It helps to work through an example. Let S be the
sentence \emph{Brian is sitting}. We will write \(⟦S⟧\) for the content
of that sentence - these weird double square brackets have become the
standard way for expressing the function from sentences to their
contents. Now we want S to be true if, and only if, Brian is indeed
sitting when S is uttered. That last bit, when S is uttered, is
important. S doesn't say that I always sit, just that I'm now sitting.
How should we incorporate that into our theory.

One approach is to build the time into the content of the utterance. On
this view \(⟦S⟧_t\) is \textbf{Brian is sitting at t}, where t is the
time the sentence was uttered. The subscript is to say that this is the
content of S at t. On this view there is no such thing as \emph{the}
content of S, since it has different contents at different times of
utterance. Then `how things are' is the entire world, from Big Bang to
Big Crunch. Call the world w, for simplicity, because we'll use it a
lot. And we'll write \(\vDash_w⟦S⟧_t\) - which you can read as saying
that the content of S is true at w, to say that this content is true.
This approach is called \textbf{eternalism}, which is an annoyingly
polysemous term in philosophy. If \(⟦S⟧_t\) is true, it is always true,
hence the eternal. It's true even if I stand up, because \(⟦S⟧_t\)
includes a reference to a time, and I stand at a later time. To
introduce one more technical term, the time is part of the
\textbf{context}; something that affects what the content of a given
utterance of the sentence is.

Another approach is to build the time into how things are. On this
approach the world does go from Big Bang to Big Crunch, but how things
are doesn't just involve a world, it involves a time too. (If you like
you can make this time metaphysically special - like on a spotlight view
- but that's not part of the linguistic theory.) The content of \(⟦S⟧\)
is just what it looks like: that Brian sits. But that very content is
true at some times and false at others. We write this as
\(\vDash_{\langle w, t \rangle}⟦S⟧\). This view is known as
\textbf{temporalism}, because the things that are true or false are
temporal entities, they change truth value over time. And to introduce
another technical term, on this theory the \textbf{index} is the pair
\(\langle w, t \rangle\). The index is what the sentence is measured up
against.

What is the difference between these views? There is a view that it's a
purely notational difference, that we shouldn't care about which is
right. But that's a minority view. Most philosophers think there is a
big issue here. The problem is that there is less agreement about what
exactly the big issue is. We want contents, the outputs of the \(⟦⟧\)
function to play some philosophical role. The problem is that there are
too many possible roles for them to play. Here are five possibilities.

\begin{enumerate}
\def\labelenumi{\arabic{enumi}.}
\tightlist
\item
  The contents of what we believe.
\item
  The contents of what we say (if you think we have independent grasp on
  this).
\item
  The contents of `that'-clauses, especially in indirect speech reports.
\item
  The things we agree or disagree with when we agree/disagree with an
  assertion.
\item
  The things we retract when we no longer endorse our former assertions.
\end{enumerate}

Sometimes these five considerations push in the same direction. So
consider the sentence \emph{I was born in Australia}. That's true when
uttered by me, and false (I think) when uttered by anyone else here. But
when I say it, you don't disagree with me. And you don't report me by
saying ``Brian said that I was born in Australia''. For reasons like
this, everyone agrees that who the speaker is is relevant to the
\textbf{content} of an utterance involving a first-person pronoun, not
to the \textbf{index}. It's not that I say something that's true
relative to me and false relative to you. It's that I say something
that's true full stop, even though had you uttered the same words, you
would have said something false. That's for the simple reason that you
would have said something different.

But sometimes these considerations come apart. In the
temporalism/eternalism debate, some considerations push towards putting
time into the index, and some into the content. And so there's a debate
here, but I'm not going to go deep into it.

Just one quick thing to note though. I've talked so far as if it is
obvious that worlds go into indices. And I think it is obvious that's
true. But not everyone agrees. Some philosophers have recently argued
that worlds should go into contexts and hence into contents. So when I
say ``There is a famous detective that lives on Baker Street'', that's
false, but the same utterance would be true in Sherlock-Holmes-world.
That's because, says orthodoxy, there is a proposition \emph{That a
famous detective lives on Baker Street} which is false in our world and
true in Sherlock-Holmes-world. But some philosophers think that just
like times go into contents (given eternalism), worlds should go into
contents too. So when I say ``There is a famous detective that lives on
Baker Street'', I express the proposition \emph{That a famous detective
lives on Baker Street in w}, where w is a name for our world. As I said,
I think this is an absurd view, but it's worth noting it is out there.
Still, most people think that indices are non-trivial, and they include
at least worlds.

A huge amount of work in philosophy, especially in the 2000s, was on
questions around the nature of context and index. In particular, these
three questions took up a lot of people's time and attention. (The first
by far the most, then the second, and then the third.)

\begin{enumerate}
\def\labelenumi{\arabic{enumi}.}
\tightlist
\item
  Which utterances are such that their truth is sensitive to context? A
  huge debate, one of the biggest of the 90s and 2000s, was about
  whether knowledge ascriptions were among them, with
  \textbf{contextualists} (about knowledge) saying that utterances like
  ``Brian knows that cars exist'' could be true in regular contexts, but
  false in `sceptical contexts'. But this was merely one manifestation
  of a discipline wide search for context-sensitivity.
\item
  If utterances are sensitive in this way, is this because the context
  affects their content, or because it affects the index they are
  evaluated against?
\item
  If it does affect the index, does it do so in a uniform way, or are
  there some sentences that interact differently with indices to others?
\end{enumerate}

Don't worry about the last - I've got it there for completeness. It's
mostly because the most important figure in this whole debate, John
MacFarlane, disagrees with the way I've set up the debate so far. And it
would be misleading to write his view out of the picture, since he is
really the most important figure in it. MacFarlane's view is
particularly motivated by issues about retraction, but I'm going to set
those issues aside, and mostly set the precise views of MacFarlane aside
as well.

For an example of what's at stake at least in questions 1 and 2, imagine
that we have with us an ancient Spartan who very helpfully speaks
English. And now let \emph{S} be the sentence ``Infanticide is wrong''.
We endorse that sentence, and the Spartan rejects it. Here are three
possible philosophical takes on that dispute.

First, we could have an invariantist approach. One of us, presumably us,
is right, and the other is wrong. This is the view that most moral
realists take, though actually a lot of views that are not particularly
realist could take it as well.

Second, we could have what I'd call a \textbf{contextualist} approach.
When we say ``Infanticide is wrong'', we speak truly. But when the
Spartan says ``Infanticide is not wrong'', he speaks truly as well. Why?
Because our sentences have different contents. The content of our
sentence is \emph{That infanticide is wrong by contemporary standards}.
And the content of the Spartan's sentence is \emph{That infanticide is
wrong by Spartan standards}. And those propositions are both true.

Third, we could have what I'd call a \textbf{relativist} approach.
(Though note MacFarlane denies this is relativism; he calls this view
`non-indexical contextualism'.) Both we and the Spartan speak truly. But
we also disagree, because we endorse the very proposition that the
Spartan rejects. How is this possible? Well, the proposition we are
disputing is only true relative to an ordered pair of a world and a
moral standard. Relative to our world/standard, it is true; relative to
the Spartan's, it is false. It's just like if I said ``There is no
famous detective living on Baker Street'', that's true, even though it
expresses the negation of a proposition that a person in
Sherlock-Holmes-world can truly express. In a figurative sense, we and
the Spartan live in different moral worlds, so our thoughts and talk can
clash, without either being wrong.

This was a very long response to an argument Williams briefly makes
against moral relativism. He thinks it is a problem for relativism that
people in isolated moral communities won't know that there are other
communities, and so won't know to relativise their moral talk. I think
this is only a problem for moral \emph{contextualism}, not moral
relativism properly speaking. The real moral relativist, in the sense of
the previous chapter, does not think that the moral utterances that
people make are in any sense about moral communities, or standards, or
anything of the sort. Rather, they think that the theory of truth for
the (simple) propositions those utterances express, involves relativity
to communities/standards. And they don't have to say that competent
speakers need realise that, any more than competent speakers have to
know anything about the physics of time in order to use tensed language.

\hypertarget{the-meta-ethical-trilemma}{%
\subsection{The Meta-Ethical Trilemma}\label{the-meta-ethical-trilemma}}

McDowell presents the following as an argument for non-cognitivism, but
I find it more helpful to consider it as a trilemma (which is of course
a trivial variation on McDowell's presentation.)

\begin{enumerate}
\def\labelenumi{\arabic{enumi}.}
\tightlist
\item
  Moral attitudes provide reasons.
\item
  Beliefs alone do not provide reasons.
\item
  Moral attitudes are beliefs.
\end{enumerate}

There are three obvious ways to respond to this, by denying each of the
three lines. McDowell calls denials of 1 \textbf{descriptivism}. Certain
kinds of reductivism will end up in this camp (if they don't end up as
error theories.) He doesn't give a label for denials of 2, but it's the
position he wants to take. And denials of 3 are
\textbf{non-cognitivism}.

There are a number of other ways to get out of the trilemma without
cleanly denying any of the three steps.

\begin{itemize}
\tightlist
\item
  As always in philosophy, it is possible to say that the argument
  equivocates. Here the most obvious equivocation is on `reason'.
  Perhaps there is a hidden equivocation between \emph{motivating
  reason} and \emph{justifying reason}. But a better way out is to say
  that there is an equivocation between \emph{decisive reason} and
  \emph{supporting reason}. Foot, for instance, denies 1 if `reason'
  means \emph{decisive reason}, and denies 2 if `reason' means
  \emph{supporting reason}.
\item
  The whole setup assumes that attitudes are the kind of thing that
  provide reasons, and without that assumption (which is a lot less
  popular now than it was 40 years ago) it isn't clear there is a
  problem here.
\item
  You could deny more than 1 of the statements. The view in my book
  entails that 1 and 2 are both false (while the view is neutral on 3).
  But that's way outside the mainstream meta-ethical views.
\item
  You could deny that there is an inconsistency here to be resolved. I
  \emph{think} you can develop a moral error theory that endorses all 3.
\end{itemize}

\newpage

\hypertarget{beliefs-and-reasons}{%
\subsection{Beliefs and Reasons}\label{beliefs-and-reasons}}

McDowell offers two reasons for endorsing 2. He isn't maximally clear
about them, but here is my best attempt at spelling them out in more
detail. The first argument starts from what I think the `hydraulic'
conception of reasons is meant to be.

\begin{enumerate}
\def\labelenumi{\arabic{enumi}.}
\tightlist
\item
  For the rational person, the reasons that motivate their actions are
  the reasons that justify their actions.
\item
  Beliefs are passive states; all it takes to believe something is to
  have a library in your head.
\item
  Motivating reasons are not passive.
\item
  So beliefs are not motivating reasons (from 2, 3).
\item
  So, for the rational person, beliefs are not justifying reasons.
\end{enumerate}

I think it's plausible to reject both 1 and 2. A rational person need
only be sensitive to their reasons, not be motivated by them. Reasons
can serve as guardrails, stopping the person when they are going to do
the wrong thing, without motivating them when they are doing the right
thing. And 2 will be false on a \textbf{functionalist} picture of
beliefs. (Though some people think this is a reason to reject
functionalism.)

The other argument I think is more interesting. This is the argument
from the idea that it must be possible to tell that a reason is a reason
from a `sideways on' perspective. That is, it must be possible to tell
that it is a reason from a neutral perspective. This view finds its
modern expression in theories that demand that norms (moral, epistemic,
whatever) are \emph{guiding}. Here's the argument I read McDowell as
giving.

\begin{enumerate}
\def\labelenumi{\arabic{enumi}.}
\tightlist
\item
  If A is a reason for X, it is possible to tell from a neutral
  perspective that A is a reason for X.
\item
  From a neutral perspective, it is not possible to tell that a belief
  (alone) is a reason for any action.
\item
  So, beliefs (alone) are not reasons for action.
\end{enumerate}

This is a really interesting argument. I agree with McDowell both that 1
is false, and that a lot of philosophical mistakes are caused by
thinking that 1 is true. One thing that you get from people believing 1
is you get a theory of reasons which is deeply tied up with formal
notions of coherence. So for something to be a reason it has to
logically entail the thing being done, or you have to be able to compute
that the thing has maximal value, or something like that. These
approaches seem false to me twice over. For one thing they licence all
sorts of terrible - but utterly coherent - views. For another they don't
even get the promised neutrality, since questions of logic, decision
theory, and game theory, and hence of coherence, are just as hard to
tell from the neutral perspective as moral questions.

There is another more contemporary view that actually agrees with
McDowell that beliefs can be reasons, but which is motivated by the same
kind of `neutral perspective' consideration. This is the view that only
thin moral beliefs can motivate. So a belief \emph{A is wrong} is a
reason not to do A, but literally no other kind of belief can be
reason-providing. Again, I think this is driven by a desire that reasons
are detectable as reasons from a neutral perspective.

\newpage

\hypertarget{non-cognitivism-and-separation}{%
\subsection{Non-Cognitivism and
Separation}\label{non-cognitivism-and-separation}}

I thought the argument in section 5 of McDowell's paper was really
fascinating, but I'm not sure it has always been fully appreciated in
the subsequent literature. (Not that I've read 100\% of it, or even
close.) The argument is an attack on the view that attitudes involving
thick concepts, like say `honest', involve two separate parts. On the
one hand, they involve a belief that the honest thing satisfies the
descriptive part of `honest'. On the other, they involve some kind of
pro-attitude towards the thing in virtue of satisfying that descriptive
component.

One way to argue against this view would be to simply say that
introspection does not reveal that the attitude is bifurcated this way.
But that's a lousy argument I think. We shouldn't trust introspection
that much. Here's a slightly better argument.

\begin{enumerate}
\def\labelenumi{\arabic{enumi}.}
\tightlist
\item
  If `honest' has an evaluative and a non-evaluative part, then someone
  who doesn't share our values could understand the non-evaluative part.
\item
  If so, they could predict how we would apply the term, even without
  sharing the values.
\item
  But someone who doesn't share our values can't predict that.
\item
  So `honest' doesn't have these two parts.
\end{enumerate}

But as Blackburn pointed out, in this argument premise 2 is false. If
`honest' means `good in virtue of H', someone could understand H and not
predict our usage of it. Here's a much more interesting argument that
turns on the fact that neither we nor outsiders can factorise `honest'.

\begin{enumerate}
\def\labelenumi{\arabic{enumi}.}
\tightlist
\item
  From the inside, the evaluative part of `honest' feels non-arbitrary;
  it doesn't seem like we just like honest things the way we might like
  a flavour of ice-cream. Call this the non-arbitrariness constraint.
\item
  The non-arbitrariness constraint requires that we can identify the
  descriptive features in virtue of which we positively evaluate them,
  and do so independently of our own evaluation.
\item
  We can only identify the descriptive parts of our own thick concepts
  independently of our own evaluations if an outsider, who did not share
  or understand our values, could identify them.
\item
  An outsider could not make this identification - this is both
  intuitive, and arguably the result of the long Wittgensteinian
  discussion in section 3.
\end{enumerate}

And now we have a problem. Line 1 says that our own attitudes involving
thick concepts satisfy the non-arbitrariness constraint, but lines 2-4
together imply that they could not.

This is why I think it matters that we can't identify the honest, or
kind, or cruel, behaviors independently of our values. Given
non-cognitivism, the only way we can pick out the kind behaviours are as
those that morally resemble, by our lights, the paradigms of kind
behaviour. But it would feel hopelessly circular to say that we admire
kind behaviour because things only fall under the concept `kind' if we
admire them. So there is a problem for non-cognitvists here.

If I were a non-cognitivist, I'd try to get out of the problem this way.
All that non-arbitrariness requires is that on any given occasion of
use, we can specify in a non-circular way why we admire the people we
admire. And that's possible even if we can't specify exactly what
kindness is. But is this enough for the non-cognitivist?

\end{document}
