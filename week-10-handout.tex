% Options for packages loaded elsewhere
\PassOptionsToPackage{unicode}{hyperref}
\PassOptionsToPackage{hyphens}{url}
%
\documentclass[
]{article}
\usepackage{amsmath,amssymb}
\usepackage{lmodern}
\usepackage{iftex}
\ifPDFTeX
  \usepackage[T1]{fontenc}
  \usepackage[utf8]{inputenc}
  \usepackage{textcomp} % provide euro and other symbols
\else % if luatex or xetex
  \usepackage{unicode-math}
  \defaultfontfeatures{Scale=MatchLowercase}
  \defaultfontfeatures[\rmfamily]{Ligatures=TeX,Scale=1}
  \setmainfont[]{SF Pro}
\fi
% Use upquote if available, for straight quotes in verbatim environments
\IfFileExists{upquote.sty}{\usepackage{upquote}}{}
\IfFileExists{microtype.sty}{% use microtype if available
  \usepackage[]{microtype}
  \UseMicrotypeSet[protrusion]{basicmath} % disable protrusion for tt fonts
}{}
\makeatletter
\@ifundefined{KOMAClassName}{% if non-KOMA class
  \IfFileExists{parskip.sty}{%
    \usepackage{parskip}
  }{% else
    \setlength{\parindent}{0pt}
    \setlength{\parskip}{6pt plus 2pt minus 1pt}}
}{% if KOMA class
  \KOMAoptions{parskip=half}}
\makeatother
\usepackage{xcolor}
\IfFileExists{xurl.sty}{\usepackage{xurl}}{} % add URL line breaks if available
\IfFileExists{bookmark.sty}{\usepackage{bookmark}}{\usepackage{hyperref}}
\hypersetup{
  pdftitle={PHIL 640: Honesty},
  pdfauthor={Brian Weatherson},
  hidelinks,
  pdfcreator={LaTeX via pandoc}}
\urlstyle{same} % disable monospaced font for URLs
\usepackage[margin=1.3in]{geometry}
\usepackage{graphicx}
\makeatletter
\def\maxwidth{\ifdim\Gin@nat@width>\linewidth\linewidth\else\Gin@nat@width\fi}
\def\maxheight{\ifdim\Gin@nat@height>\textheight\textheight\else\Gin@nat@height\fi}
\makeatother
% Scale images if necessary, so that they will not overflow the page
% margins by default, and it is still possible to overwrite the defaults
% using explicit options in \includegraphics[width, height, ...]{}
\setkeys{Gin}{width=\maxwidth,height=\maxheight,keepaspectratio}
% Set default figure placement to htbp
\makeatletter
\def\fps@figure{htbp}
\makeatother
\setlength{\emergencystretch}{3em} % prevent overfull lines
\providecommand{\tightlist}{%
  \setlength{\itemsep}{0pt}\setlength{\parskip}{0pt}}
\setcounter{secnumdepth}{-\maxdimen} % remove section numbering
\linespread{1.18}
\ifLuaTeX
  \usepackage{selnolig}  % disable illegal ligatures
\fi

\title{PHIL 640: Honesty}
\author{Brian Weatherson}
\date{November 15, 2021}

\begin{document}
\maketitle

\hypertarget{priority-questions}{%
\subsection{Priority Questions}\label{priority-questions}}

We describe both people and actions as honest. One interesting
philosophical question concerns priority. Both of these things are true,
but which of them are explanatory?

\begin{enumerate}
\def\labelenumi{\arabic{enumi}.}
\tightlist
\item
  Honest actions are actions that honest people would be disposed to do.
\item
  Honest people are people who are disposed to do honest actions.
\end{enumerate}

Miller says that 1 is explanatory and 2 is not; what makes something an
honest action is that it is something an honest person would do. But I
didn't really see an argument for this, and I could see a case made for
the other side.

Indeed, I wasn't sure that Miller's own theory supported this direction
of explanation. It's easy to at least gesture at what makes an action
honest. Something about being truthful, or pro-truth, or something like
that. And then you can gesture at what makes a person honest - they do
those kinds of actions. (Perhaps typically, or perhaps in the important
cases.) What's the story in the reverse direction?

Anyway, this is (I think) what he means when he says the virtue is prior
to the actions. And I don't know what the argument is for this priority
claim, even though it seems like a fairly important point to me.

\hypertarget{honest-actions}{%
\subsection{Honest Actions}\label{honest-actions}}

Miller gives an interesting list of five kinds of behavior that he
thinks involve honesty. And we should be interested in (at least) three
questions about this list.

\begin{enumerate}
\def\labelenumi{\arabic{enumi}.}
\tightlist
\item
  Is it a correct description of how the term `honest' is used in
  English?
\item
  Which of these kinds are such that \emph{every} instance of them is an
  instance of dishonesty, and which of them are such that \emph{typical}
  instances of them are instances of dishonesty?
\item
  What do these things have in common?
\end{enumerate}

That said, here is the list.

\begin{itemize}
\tightlist
\item
  Lying
\item
  Stealing
\item
  Cheating
\item
  Promise Breaking
\item
  Deceiving
\end{itemize}

My view is that the first and the fifth are clearly instances of
dishonesty, and the third might be - depending on what you mean by
cheating. But I'm really not sure that stealing or promise breaking are.
We'll come back to this when we look at Miller's explanation of why they
are dishonest.

On cheating, it's perhaps stepping back to think about what we call
cheating. Not all kinds of rule violations are cheating. A football
player (in any code of football) who is offside isn't cheating, even if
they break a rule. Someone who steps out of bounds while corralling a
ball isn't cheating. What we call cheating, as opposed to mere
rule-violation, I think involves some element of deception. So it isn't
really a separate category.

That said, I'm not sure the later discussion of cheating, when Miller
discusses steroids in baseball, really gets to the heart of the matter.
I don't think the steroids taker is representing themselves as not using
steroids. (Ignore cases where they expressly say they aren't using
steroids - those obviously are cases of deception!) Imagine a league
where this kind of cheating is rampant. (As it was in baseball until
recently!) It seems weird to say that everyone is implicitly
misrepresenting their status. The kind of representation Miller has in
mind here only works if the typical baseball player is not cheating in
this way.

There are also hard questions about what it takes for something to be
`natural'. Does someone who uses a hyperbaric chamber to speed recovery
from injury count as using natural methods? (It might be relevant here
to note that one of the main benefits of steroids in baseball just is
speeding recovery time.) I'm not sure how steroids are unnatural but the
hyperbaric chamber is natural. It's just that we have a rule against one
but not the other.

So here are three questions I'd like us to have on the table. (And if
this was a more general philosophy of sport class we'd spend
\textbf{way} longer on each.)

\begin{enumerate}
\def\labelenumi{\arabic{enumi}.}
\tightlist
\item
  What is the difference between cheating and rule-breaking?
\item
  Must cheating, as opposed to rule-breaking, involve deception?
\item
  Must cheating involve misrepresentation? (Subsidiary question: what
  facts were the Houston Astros distorting/misrepresenting when they
  cheated?)
\end{enumerate}

\hypertarget{the-honest-thief}{%
\subsection{The Honest Thief}\label{the-honest-thief}}

We're not going to go over the plot of either the Dostoyevsky story or
the Liam Neeson movie - though if someone wants to discuss either we
can. (Though preferably with warning - I haven't read/seen either!) But
I do think it's interesting that we do have the phrase `honest thief' in
English (and I guess an equivalent in Russian). I'll come back to what
Miller says about this case, but I wanted to flag two questions.

\begin{enumerate}
\def\labelenumi{\arabic{enumi}.}
\tightlist
\item
  Is there something about thieving that is distinctly connected to
  dishonesty, in a way that other forms of immorality are not?
\item
  If so, what might that connection be?
\end{enumerate}

For what it's worth, my guess is that there is merely a statistical
connection here. A good chunk of thieving involves deception of some
kind or other. But maybe there is something different to it.

\hypertarget{honest-promise-breaker}{%
\subsection{Honest Promise Breaker}\label{honest-promise-breaker}}

The first thing to do here is to think about the same kind of questions
that we discussed under thieving.

\begin{enumerate}
\def\labelenumi{\arabic{enumi}.}
\tightlist
\item
  Is there something about promise breaking that is distinctly connected
  to dishonesty, in a way that other forms of immorality are not?
\item
  If so, what might that connection be?
\end{enumerate}

Here I'm even less convinced that there is a distinctive connection to
dishonesty. A promises B that they'll pick up B from the airport. A
forgets to set their alarm, sleeps in, and hence is not in time to pick
B up from the airport. They've broken a promise and done something
wrong. Have they acted dishonestly \emph{at all}?

There is something dishonest about making a promise that one never
intends to keep. And such a promise I guess typically ends up getting
broken. But it's not the breaking of the promise that seems to be the
dishonesty in that case; it's the flawed promise making. I'm not sure
that there is any connection at all. But I'd be interested in knowing if
others thought differently.

\hypertarget{honesty-and-forthrightness}{%
\subsection{Honesty and
Forthrightness}\label{honesty-and-forthrightness}}

I think as I use the terms `honest' and `dishonest', there is an
asymmetry between them concerning forthrightness. Consider the following
situation.

\begin{itemize}
\tightlist
\item
  It is common ground that A wants to know the answer to some question
  Q?
\item
  B knows the answer, and knows that the answer reflects somewhat badly
  on B.
\item
  There is no reason for anyone to suspect that B knows the answer; B
  not saying anything about Q? isn't any kind of signal that B doesn't
  know the answer, because no one thinks they know.
\item
  B is not asked whether they know, so they don't have to say that they
  don't know.
\end{itemize}

In such a case I think it would be honest of B to reveal the answer. I
might even say it was \emph{very honest}. But I wouldn't think it was
dishonest to not reveal it. To expressly say one didn't know, or even to
give people evidence that one didn't know, might be dishonest. But
simply not confessing to a misdeed isn't, \emph{I think}, a kind of
dishonesty.

\begin{itemize}
\tightlist
\item
  Do you agree with those claims, i.e., that confessions to minor
  misdeeds can amount to (very) honest behaviour, but non-confessions
  are not in themselves dishonest?
\item
  If so, what could explain the asymmetry here?
\item
  Also if so, under what circumstances can mere silence amount to
  dishonesty? Are there any such circumstances?
\end{itemize}

\hypertarget{honesty-and-immorality}{%
\subsection{Honesty and Immorality}\label{honesty-and-immorality}}

First, a big picture question.

\begin{itemize}
\tightlist
\item
  Is all immorality dishonest?
\end{itemize}

I think the answer is surely no. A dislikes B. So A walks up to B one
day, and punches him hard in the nose. This is bad, you shouldn't punch
people. But it is hard to say that it is dishonest behaviour. So not all
immoral behaviour is dishonest.

But note now that we think an argument Miller makes (twice) has to fail.
Here's how he explains how thieving and promise breaking are acts of
dishonesty: they amount to misrepresenting the normative features of the
world. The thief represents themselves as having the right to steal the
goods. And that's wrong, so it's a misrepresentation. But A
misrepresents themselves as having the right to punch B in the nose. And
that's wrong, so it's a misrepresentation. So this kind of implicit
normative misrepresentation can't be enough to amount to dishonesty.

This is why I'm a little sceptical that Miller's story about why
thieving and promise breaking are dishonest. It looks like an argument
that overgenerates to `show' that assault is dishonest.

\hypertarget{is-honesty-always-good}{%
\subsection{Is Honesty Always Good?}\label{is-honesty-always-good}}

I don't have a lot to add to what Miller says on this point, save to
note an interesting historical point. The main example that gets used
here is, as you might guess if you've taken any philosophy classes at
all, the murderer at the door. And we get the following questions.

\begin{enumerate}
\def\labelenumi{\arabic{enumi}.}
\tightlist
\item
  Is it good to lie to the murderer at the door?
\item
  Is it honest to lie to the murderer at the door?
\end{enumerate}

If you think, as I do, that the answers to these questions are `yes' and
`no', then it follows that it isn't always good to be honest. That's
easy enough. But I wanted to mention a couple of things about the
example.

For one thing, although we normally credit the example to Kant, it's
really due to Benjamin Constant, and in particular it's from chapter 8
of his
\href{http://classiques.uqac.ca/classiques/constant_benjamin/des_reactions_politiques/reactions_politiques.doc\#des_réactions_chap_08}{Des
réactions politiques}. Here is one key passage, with an automated
translation.\_

\begin{quote}
Le principe moral, par exemple, que dire la vérité est un devoir, s'il
était pris d'une manière absolue et isolée, rendrait toute société
impossible. Nous en avons la preuve dans les conséquences très directes
qu'a tirées de ce principe un philosophe allemand, qui va jusqu'à
prétendre qu'envers des assassins qui vous demanderaient si votre ami
qu'ils poursuivent n'est pas réfugié dans votre maison, le mensonge
serait un crime.

The moral principle, for example, that telling the truth is a duty, if
taken in an absolute and isolated way, would make any society
impossible. We have proof of this in the very direct consequences of
this principle by a German philosopher, who goes so far as to claim that
against murderers who would ask you if your friend they are pursuing is
not a refugee in your house, lying would be a crime.
\end{quote}

Now here are two interesting things about this. Constant's book is
published (and I guess written) in 1796. Kant's reply, which is only 2
pages long and starts with a long quote from Constant, comes out the
following year. And Constant isn't really interested in lying. What he's
interested in is how to deal with political injustice. Which is a very
interesting question for a leading French intellectual to be writing
about in the immediate aftermath of the Reign of Terror.

Philosophers sometimes kind of ignore the dates on things they are
writing about. Oh yes, Kant wrote these things in the 1780s and these
things in the 1790s and we have to know the order of these because they
reflect how Kant's view developed over time. That's true, but it's also
maybe relevant that during that time the world got turned upside down
and maybe that should be taken seriously in thinking about why those
developments may have taken place.

There are a couple of things that could follow from this. Geneviève
Rousselière has argued, in
\href{https://journals.sagepub.com/doi/abs/10.1177/1474885115588100?journalCode=epta}{On
Political Responsibility in Post-Revolutionary TImes: Kant and
Constant's Debate on Lying} that the whole debate should be read as an
allegory for their differing views about political philosophy and the
Revolution, and it isn't really about lying at all. I don't know if
that's right, but it is more plausible than the other interpretations
I've seen of Kant's really bizarre little paper.

But the other thing to remember is that the murderer at the door example
isn't really the outlandish example that it's sometimes taken to be. For
a few years there, people turning up at a door and asking for where a
person who they are looking to kill might be was a kind of everyday
occurrence. Now sure they weren't looking to kill them on the spot,
merely drag them in front of the Revolutionary Tribunal. And the
Tribunal actually acquitted a few more people than you might expect from
the standard popular history. But still, describing the agents of the
Revolution as murderers at the door wouldn't be the most outlandish
description ever. From this perspective, the Constant-Kant debate is a
modernised version of the debate in \emph{Crito}, which I think makes it
a bit more interesting than the way it is usually presented.

Constant himself is an interesting figure. He's probably less famous
these days than his long-time partner, Germaine de Staël, and that's
probably fair. But he's still an interesting representation of a kind of
liberalism that's much more popular now than it was at the time. And to
give a sense of how widely known he was, he gets name-checked near the
start of Marx's \emph{18th Brumaire} with the expectation that the
reader will know precisely who he's talking about. But some of his
significant works, including the one that Kant engages with in the
murderer at the door paper, aren't even translated into English.
Automated translations can help a bit, but probably human ones would be
better. I was fascinated, for example, by this claim, at the end of a
long discussion about the right way to understand the duty to tell the
truth.

\begin{quote}
C'est une idée peut-être neuve, mais qui me paraît infiniment
importante, que tout principe renferme, soit en lui-même, soit dans son
rapport avec un autre principe, son moyen d'application.

It is perhaps a new idea, but it seems infinitely important to me, that
every principle contains, either in itself or in its relationship with
another principle, its means of application.
\end{quote}

That's a really fascinating of some ideas we find playing a central role
in Wittgenstein's \emph{Investigations}. Anyway, this has been a
long-enough digression, even if we did end up back somewhere from
earlier in the course. Let's get back to honesty.

\hypertarget{instrumental-honesty}{%
\subsection{Instrumental Honesty}\label{instrumental-honesty}}

This is something we could ask about all virtues, but let's ask it about
honesty. If someone consistently manifests all the signs of a virtue,
but they mostly do this for instrumental reasons, do they really have
the virtue? Let's think about two cases.

\begin{itemize}
\tightlist
\item
  A shopkeeper believes that a reputation for honesty is crucial for the
  success of his shop, and he believes that he isn't skillful enough to
  fake such a reputation. The only way he'll be seen to be honest is if
  he acts honestly. So that's how he acts, consistently.
\item
  A theist believes that a reputation for honesty is crucial for getting
  to heaven, and she believes that she isn't skillful enough to fake
  such a reputation. (She doesn't really think anyone could fake out
  God.) The only way she'll be seen to be honest by God is if she acts
  honestly. So that's how she acts, consistently.
\end{itemize}

Again, a few questions.

\begin{enumerate}
\def\labelenumi{\arabic{enumi}.}
\tightlist
\item
  Is there any difference in virtue between these two people?
\item
  If not (as I think), are they both honest, or both dishonest?
\item
  If there is a difference, what makes the cases different?
\end{enumerate}

I used to think that the two cases were alike and they were both not
really cases of honesty. But as I've got older and mellower, I've
started to think that consistently acting out a virtue might just be
enough to have it. These both seem like honest people to me. They might
have gotten to honesty in a slightly sub-optimal way, but we all get to
virtues in weird and roundabout ways.

But what do you think? Does honesty require having a really deep
commitment to honesty? Or is doing it for the rewards, either in this
life or the next, good enough?

Miller doesn't discuss this, but we could ask the same about the reverse
case. Imagine someone who really wants to get ahead in a particular
business. And they want this because it will mean they have more money
that they can use to help raise their children, donate to charity, etc.
But their manager really values dishonesty - he thinks that it's crucial
to succeed in this business. So the person has to develop a reputation
for dishonesty. And they don't think they can fake it - they have to
actually act dishonestly. So that's what they do - feeling bad about it
because they value honesty; they just value money for their children and
the needy a bit more. Is this kind of instrumental dishonesty enough to
make them have the vice of dishonesty? If so, is this an asymmetry
between honesty and dishonesty?

\hypertarget{honest-mistakes}{%
\subsection{Honest Mistakes}\label{honest-mistakes}}

Miller leaves it as an open question whether honesty is a matter of
distorting the facts as they are, or the facts as they appear. But this
seems like an easy question to me. We have the concept of an honest
mistake. That surely means a kind of case where someone says something
false - the mistake part - but is being honest because they are saying
what they believe. That is, you can't be dishonest while being
completely sincere; at worst you're making an honest mistake.

I'm not normally this categorical about disputed philosophical
questions, but I really don't see what the rival view could be here.

\hypertarget{threshold-concepts}{%
\subsection{Threshold Concepts}\label{threshold-concepts}}

Miller's notion of a threshold concept was, I thought, a bit confusing.
And I really wasn't clear on what he meant by saying that both honesty
and dishonesty are threshold concepts. This is in part because he seemed
to run together two notions here. Both of them are interesting, but they
aren't the same notion.

The first notion concerns the mediocre. He wants to say that they are
(determinately) neither honest nor dishonest. So the person who
occasionally lies to get out of a jam, every so often engages in petty
theft (office supplies and the like) or makes a promise they don't have
any intent of keeping, but really does all these things less often than
the median human, is neither honest nor dishonest. The point here is to
deny that honest and dishonest merge into each other. Just like you
might think (I guess I think) that there are shades of purple that are
determinately neither red nor blue, there are dispositions vis a vis
honesty that are neither on honest nor dishonest. I'm not 100\% sure I
agree, but it sounds plausible enough.

The second notion concerns the multi-dimensionality of the notion of
honesty. Miller thinks it has these five components that he lists. I
don't really think they are all part of honesty, but I do think
forthrightness is part of honesty, and generally I agree that it's
multidimensional. Here the notion of a threshold concept seems to be
that if you're doing really badly on any dimension, that defeats calling
you honest. And I guess I agree with that. If I thought non-deceptive
theft was a kind of dishonesty, I would say that a frequent thief is not
honest, no matter how few lies they tell etc.

But here's where the problems arise. \textbf{Dishonesty} doesn't have
anything like the same logical structure. Someone who is really bad on
one dimension of honesty/dishonesty is dishonest, even if they are
really good on the other dimensions. If \emph{being a threshold concept}
requires that you score highly enough on every dimension in order to
satisfy the concept, then dishonesty is not a threshold concept. And
it's misleading to talk about this being a case of symmetry between
honesty and dishonesty. One of them, the positive one, requires a high
enough score on every dimension. The other one can be satisfied by being
dishonest enough in just one way.

To some extent this is terminological. But I think it's important to
note that the virtue and the vice really don't have the same logical
structure here, and talking as if they do just obscures things.

\hypertarget{psych-problem-1-fallacy-of-composition}{%
\subsection{Psych Problem 1: Fallacy of
Composition}\label{psych-problem-1-fallacy-of-composition}}

Much of the paper is given over to empirical evidence that people are
not in fact honest. And much of that was interesting, but occasionally
not as strong a support for Miller's pessimistic conclusions as he
presented. The reason for this is independently interesting enough to go
over.

The fallacy of composition is when one attributes to the group things
that are really just true of a small number of the parts. The example
that I see most commonly concerns voting behaviour. It's common to hear
people say that such and such community changed their mind when the
majority swung between one political group and another. Now in a
democracy this change in majority might be really significant
practically. But talking about it as a change in mind of the group is
potentially misleading - it might have been 10\% or less of the group
that actually changed their mind for the majority to change.

Some of Miller's examples seemed to be like that. You could get a
reported average of 6.6 heads in the coin flipping experiment by (a) a
small bias, either by luck or design, in the coins (and note that they
did get 5.2 not 5 in a control group), plus (b) a small-to-moderate
sized group who went overboard with dishonesty. You don't need
particularly widespread dishonesty to get the results.

Now to be fair, some of the tests did seem to require (or directly test
for) dishonesty on a person-by-person basis. But this kind of inference
from properties of salient individuals to properties of the group is a
bit too common in use of empirical data, especially by
non-psychologists. And you should always watch out for it.

\hypertarget{psych-problem-2-replication}{%
\subsection{Psych Problem 2:
Replication}\label{psych-problem-2-replication}}

One big problem with using psych papers in philosophy, especially papers
from before 2012 or so, is that the results of individual experiments
don't always replicate. Just why this is so is a big and controversial
question, but it's something you have to always be careful for.

Now Miller cites so many papers, from so many different authors, that it
is unlikely that all of them will have replication problems. (Even
there, you'd want to know what the connections are between the authors.
If they all went to the one grad school, there could be systematic
problems to watch out for.) And I suspect many of these would do fine,
though it would always be nice to know.

But note that one of them, which did have a bit of uptake outside of
academia, totally failed to replicate. This was the experiment that
claimed people were more honest if you put the ``Sign here to promise
that you're honest'' at the start rather than the end. And it turns out,
that doesn't really work. Here's the abstract for
\href{https://www.pnas.org/content/117/13/7103}{Signing at the beginning
versus at the end does not decrease dishonesty}

\begin{quote}
In 2012, five of the current authors published a paper in PNAS showing
that people are more honest when they are asked to sign a veracity
statement at the beginning instead of at the end of a tax or insurance
audit form. In a recent investigation, across five related experiments
we failed to find an effect of signing at the beginning on dishonesty.
Following up on these studies, we conducted one preregistered,
high-powered direct replication of experiment 1 of the PNAS paper, in
which we failed to replicate the original result. The current paper
updates the scientific record by showing that signing at the beginning
is unlikely to be a simple solution for increasing honest reporting.
\end{quote}

Some of the non-replications, I suspect, are due to dishonesty in the
original papers. But they don't normally get critiqued by the same
authors. (Though there are some prominent cases where people accuse
their past co-authors of dishonesty.) Sometimes there are honest
mistakes, or bad luck. And if I had to guess, I'd say that's what
happened here.

I'm just noting this because the 2012 paper is one of the ones that
Miller cites. It's good to bring empirical data to the question of how
people actually are, but you've got to be careful about how good the
data are.

\end{document}
