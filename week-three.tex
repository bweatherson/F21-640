% Options for packages loaded elsewhere
\PassOptionsToPackage{unicode}{hyperref}
\PassOptionsToPackage{hyphens}{url}
%
\documentclass[
]{article}
\usepackage{amsmath,amssymb}
\usepackage{lmodern}
\usepackage{ifxetex,ifluatex}
\ifnum 0\ifxetex 1\fi\ifluatex 1\fi=0 % if pdftex
  \usepackage[T1]{fontenc}
  \usepackage[utf8]{inputenc}
  \usepackage{textcomp} % provide euro and other symbols
\else % if luatex or xetex
  \usepackage{unicode-math}
  \defaultfontfeatures{Scale=MatchLowercase}
  \defaultfontfeatures[\rmfamily]{Ligatures=TeX,Scale=1}
  \setmainfont[BoldFont = SF Pro Text Medium]{SF Pro Text Light}
  \setmathfont[]{Fira Math}
\fi
% Use upquote if available, for straight quotes in verbatim environments
\IfFileExists{upquote.sty}{\usepackage{upquote}}{}
\IfFileExists{microtype.sty}{% use microtype if available
  \usepackage[]{microtype}
  \UseMicrotypeSet[protrusion]{basicmath} % disable protrusion for tt fonts
}{}
\makeatletter
\@ifundefined{KOMAClassName}{% if non-KOMA class
  \IfFileExists{parskip.sty}{%
    \usepackage{parskip}
  }{% else
    \setlength{\parindent}{0pt}
    \setlength{\parskip}{6pt plus 2pt minus 1pt}}
}{% if KOMA class
  \KOMAoptions{parskip=half}}
\makeatother
\usepackage{xcolor}
\IfFileExists{xurl.sty}{\usepackage{xurl}}{} % add URL line breaks if available
\IfFileExists{bookmark.sty}{\usepackage{bookmark}}{\usepackage{hyperref}}
\hypersetup{
  pdftitle={PHIL 640: Week Three- Williams},
  pdfauthor={Brian Weatherson},
  hidelinks,
  pdfcreator={LaTeX via pandoc}}
\urlstyle{same} % disable monospaced font for URLs
\usepackage[margin=1.5in]{geometry}
\usepackage{graphicx}
\makeatletter
\def\maxwidth{\ifdim\Gin@nat@width>\linewidth\linewidth\else\Gin@nat@width\fi}
\def\maxheight{\ifdim\Gin@nat@height>\textheight\textheight\else\Gin@nat@height\fi}
\makeatother
% Scale images if necessary, so that they will not overflow the page
% margins by default, and it is still possible to overwrite the defaults
% using explicit options in \includegraphics[width, height, ...]{}
\setkeys{Gin}{width=\maxwidth,height=\maxheight,keepaspectratio}
% Set default figure placement to htbp
\makeatletter
\def\fps@figure{htbp}
\makeatother
\setlength{\emergencystretch}{3em} % prevent overfull lines
\providecommand{\tightlist}{%
  \setlength{\itemsep}{0pt}\setlength{\parskip}{0pt}}
\setcounter{secnumdepth}{-\maxdimen} % remove section numbering
\linespread{1.18}
\ifluatex
  \usepackage{selnolig}  % disable illegal ligatures
\fi
\newlength{\cslhangindent}
\setlength{\cslhangindent}{1.5em}
\newlength{\csllabelwidth}
\setlength{\csllabelwidth}{3em}
\newenvironment{CSLReferences}[2] % #1 hanging-ident, #2 entry spacing
 {% don't indent paragraphs
  \setlength{\parindent}{0pt}
  % turn on hanging indent if param 1 is 1
  \ifodd #1 \everypar{\setlength{\hangindent}{\cslhangindent}}\ignorespaces\fi
  % set entry spacing
  \ifnum #2 > 0
  \setlength{\parskip}{#2\baselineskip}
  \fi
 }%
 {}
\usepackage{calc}
\newcommand{\CSLBlock}[1]{#1\hfill\break}
\newcommand{\CSLLeftMargin}[1]{\parbox[t]{\csllabelwidth}{#1}}
\newcommand{\CSLRightInline}[1]{\parbox[t]{\linewidth - \csllabelwidth}{#1}\break}
\newcommand{\CSLIndent}[1]{\hspace{\cslhangindent}#1}

\title{PHIL 640: Week Three- Williams}
\author{Brian Weatherson}
\date{September 20, 2021}

\begin{document}
\maketitle

\hypertarget{big-picture}{%
\subsection{Big Picture}\label{big-picture}}

Most writers we look at will fall neatly enough into one of two
categories.

\begin{enumerate}
\def\labelenumi{\arabic{enumi}.}
\tightlist
\item
  They take thick concepts seriously, i.e., they think that they are
  philosophically consequential, and one of those consequences is that
  they undermine philosophical theories which rely on (anything like) a
  fact/value distinction.
\item
  They think that thick concepts are at most an interesting bit of
  language to be explained away, and do not threaten the fact/value
  distinction that they rely on in their theories.
\end{enumerate}

Most of the people in category 2 rely on a fact/value distinction either
in their metaphysical picture, or their picture of mind and language.
Occasionally there are people who have a fact/value distinction in their
epistemology but not in the metaphysics or theory of content. (Possibly
that's literally just me.)

Williams, despite being the person everyone cites as being the
historically central figure in the discussion of thick concepts, doesn't
fall neatly into either camp. He does have a fact/value distinction - or
at least something very much like it. So you might think he'd be in
category 2. But in fact he thinks thick concepts are very significant to
philosophy - though in his case the significance is a distinct kind
again. He thinks that thick concepts are particularly significant
ethically.

But he also thinks that a kind of ethical relativism is true, and he
does not think that any kind of scientific relativism is true. So he
needs some kind of distinction. And when thick concepts come in, they
actually seem like a kind of problem for his view - since they suggest
that the kind of objectivism about ethics he disparages might be
defensible.

\hypertarget{the-distinction}{%
\subsection{The Distinction}\label{the-distinction}}

I've been calling it roughly the fact/value distinction, but Williams
spends a lot of ink on why that's not, in his opinion, the right way to
conceive of it.

It's not the fact/value distinction because not all questions about
value are on the right hand side of the distinction. In particular, a
lot of questions in aesthetics are about value, but they are not (I
think) on the right hand side of the distinction for Williams. He also
thinks `fact' isn't right for the left-hand side, but I couldn't really
see why. It's possible that he's assuming some kind of minimalism about
facts, so there can be ethical facts, it's just that they aren't
scientific facts.

It's not the theoretical/practical distinction either. Not all questions
about what to do are ethical questions. Children fighting over a cake
are engaged in a practical dispute, but not an ethical one. (Though
either may try to turn it into an ethical one by invoking principles of
justice.) And I think Williams wants to say that not all factual
questions are theoretical ones either. There is something funny about
describing my wondering where my car keys as a theoretical inquiry,
though I also don't think we go seriously wrong by this way of speaking.

What he does ultimately think is that it is a distinction between
objective and non-objective fields. And the explanation is interestingly
epistemological. A field is objective if we can expect that there will
be convergence in people's opinions, and this convergence is explained
by the opinions being correct. Or at least that's the initial theory -
I'm not sure how consistently Williams sticks to it. But even this
theory is worth spelling out a bit, and it's worth first taking a step
back to something that is very familiar.

\hypertarget{williamss-background-theory-of-knowledge}{%
\subsection{Williams's Background Theory of
Knowledge}\label{williamss-background-theory-of-knowledge}}

So as many of you know, contemporary epistemology starts with the
publication of a three page paper in 1963: Edmund Gettier's ``Is
Justified True Belief Knowledge?'' Gettier wanted to reject the follow
class of theories of knowledge.

S knows that p just in case these three conditions are met:

\begin{enumerate}
\def\labelenumi{\arabic{enumi}.}
\tightlist
\item
  S believes that p.
\item
  This belief is true.
\item
  This belief is `justified.'
\end{enumerate}

You can treat the third as a placeholder for any notion that assigns
positive normative status to the belief. So we could just as easily say
that the belief is sensible, or reasonable, or held for good reasons, or
etc. And the worry is that cases where 2 and 3 are both true, but they
are true for completely unrelated reasons, don't seem like knowledge. It
wasn't immediately realised, but there are cases of this form, actually
used for refuting roughly this theory, from a few places in the history
of philosophy. The following two examples are both taken from Ichikawa
and Steup (2013).

\begin{quote}
Imagine that we are seeking water on a hot day. We suddenly see water,
or so we think. In fact, we are not seeing water but a mirage, but when
we reach the spot, we are lucky and find water right there under a rock.
Can we say that we had genuine knowledge of water? The answer seems to
be negative, for we were just lucky. (From the 8th century North Indian
philosopher Dharmottara)
\end{quote}

\begin{quote}
Let it be assumed that Plato is next to you and you know him to be
running, but you mistakenly believe that he is Socrates, so that you
firmly believe that Socrates is running. However, let it be so that
Socrates is in fact running in Rome; however, you do not know this.
(From the 14th century Italian philosopher Peter of Mantua)
\end{quote}

It is important in these cases to assume that the false belief is
justified, reasonable, held for good reasons, whatever you think should
go in 3. But it should be possible to have cases that are like that.

So after 1963 there were a flood of ideas about how to either add a
fourth condition to this account of knowledge, or change the 3rd
condition in a way that made it immune to these problems. The general
idea was that we had to somehow link the 2nd and 3rd conditions, and I
think something like this is basically right.

Williams endorses a specific version of this kind of strategy, the one
defended by Robert Nozick in his \emph{Philosophical Explanations}
(1981). Simplifying a lot, Nozick wanted to replace the third condition
with the following counterfactual.

\begin{itemize}
\tightlist
\item
  If p were not true, S would not have believed it.
\end{itemize}

This feels like it gets a lot of everyday cases right. We say when this
condition is met that S's belief \textbf{tracks} the truth. And this is
a good property for beliefs to have. Indeed, you can see a lot of
safeguards we put on scientific experiments as aims to make sure the
results of experiments, and hence of experimenters' beliefs, track the
truth about the experimental situation.

But it can't be right as a theory of knowledge. A lot of problems were
raised for it almost immediately by Saul Kripke, who naturally didn't
publish these problems for three decades. (See Ichikawa and Steup (2013)
for more about Kripke's lectures.) Here's one kind of case that seems to
be devastating for this view. Imagine that I'm currently completely
sober, but when I drink my judgment goes awry in all sorts of ways. Then
the following claims could both be true.

\begin{enumerate}
\def\labelenumi{\arabic{enumi}.}
\tightlist
\item
  I know that I'm sober enough to drive.
\item
  If I were not sober enough to drive, I would still believe that I'm
  sober enough to drive.
\end{enumerate}

And there are many more problems where that comes from. While Nozick's
theory was popular for a while, it is now as universally accepted as
anything ever is accepted in philosophy that it simply fails.\footnote{See,
  there is philosophical progress!}

Is it a problem that Williams relies on a false theory here? I don't
think it is. Because while Nozick's theory fails, he doesn't need that
theory to be true, and there are near enough theories that are true.

As he notes at a few times, Williams ultimately isn't relying on
theories of knowledge here. He wants an account of what objectivity is,
and he allows that people can have knowledge of the non-objective. So he
could just drop the references to Nozick, or treat Nozick's view as at
best an inspiration.

Or he could adopt the kind of theory that virtue epistemologists such as
Ernest Sosa and Linda Zagzebski have endorsed. They think that the core
extra condition is something like that what justifies the belief is in
part the truth of the belief. We'll say more on this when we do more
virtue epistemology.

\hypertarget{scientific-realism}{%
\subsection{Scientific Realism}\label{scientific-realism}}

I found the discussion of `science' in chapter 8 very tough going, in
part because I don't know what he even means by science. It's this very
classically English philosophy of science that doesn't get its hands
dirty by engaging with any actual scientific work. When you look at
quantum physics, do you see one step of convergence after another? Or
macro-economics?

Williams's theory is that we can accept a kind of realism in science
because in science we have theorists whose beliefs converge, and the
convergence is explained by their truth. It seems easy pickings for a
dedicated scientific anti-realist, like say Fraassen (1980), to reply to
this. Maybe we get convergence because theories do better and better at
predicting the macroscopic phenomena. But since there are so many
possible theories that predict the macroscopic phenomena, and since so
many theories that approximately predicted macroscopic phenomena in the
past have been rejected, that's no reason to believe in the truth of
contemporary theories.

\hypertarget{what-is-objectivity-for-williams}{%
\subsection{What is Objectivity for
Williams?}\label{what-is-objectivity-for-williams}}

Throughout chapter 8 there are a number of overlapping ideas that come
up as being criteria for a field being objective. And I wanted to lay
some of them out, because they didn't all seem to be the same to me.

\begin{description}
\tightlist
\item[Convergence on Reality]
It is reasonable to expect that people's beliefs will converge, and that
what will explain the convergence is that the beliefs are true.
\end{description}

This is the original, and I think official, definition of objectivity.
As I said in the previous section, I think it's actually a fairly bold
claim to think that this is true of science. But it isn't the only thing
Williams puts forward.

\begin{description}
\tightlist
\item[Convergence on Justification]
It is reasonable to expect that people's beliefs will converge, and and
that what will explain the convergence is the same thing as what
justifies the people in having their belief.
\end{description}

This is closer to what we see when we get onto the stuff about thick
concepts later in the chapter. It's how I interpret the slogan he
repeats a few times ``What explains is what justifies.'' And while it's
similar to the first one, it isn't the same. Various anti-realist
approaches to science, especially those driven by anti-metaphysical
views, might think that scientists' views will converge, and that this
convergence will be explained by the same phenomena (especially
observational data) that justifies the scientists in their beliefs. But
you can say all this while still thinking it not clear whether
scientific theories are even the right kinds of things to be true.

\begin{description}
\tightlist
\item[Co-Tenability]
Either one of the convergence criteria are met, or the end state will
not be convergence, but a partition of space into people with different
views. But if the latter obtains, then there will be an external
viewpoint from which any person in any part of the partition can view
the other cells as correct from their standpoint.
\end{description}

I think, think, that this is what is going on in the color case. The
discussion of color is very confusing. It reads to me like there is a
small set of theories that are acceptable-in-Oxford-in-1985, and there
is a tacit assumption somewhere that the disjunction of these is
correct. There are various moves of the form \emph{If not this, then
that} which I couldn't make sense of really on their own (because `this'
and `that' are defined in terms of those theories) but also couldn't see
why they were the only alternatives.

But as best as I can make out, what's important to Williams is that the
people involved can see each other as being part of a broader system,
and the system justifies (and makes true?) their own beliefs, but also
justifies (and makes true?) the beliefs of the others.

I'd like to say exactly how this is meant to work in the case of color,
but I honestly can't tell precisely what theory of color Williams is
working with here. I can't even really tell if he takes the canonical
sentences to be things like \emph{That's green}, or \emph{Those two
things are the same color}. These raise quite different issues. You
probably can't use the word `green' unless you have some acquaintance
with our color practices. But you can say that things are the same
color. And you can say that, or deny it, in cases where you have very
different metamers to typical humans. (There is also the tricky fact
that there is huge variation in color vision within humans, some of it
surprisingly correlated with biological sex, so what `our' practice is
here is a bit elusive.)

There are two other things that get mentioned at various points, neither
of which seem exactly like any of the three things said so far.

\begin{description}
\tightlist
\item[Explanation of Error]
An explanation of why most of us converge should also explain why some
people make errors, and what those errors consist in.
\item[Repeating Term in Explanation]
The explanation of convergence shouldn't change the subject, and could
use the term whose convergence we're explaining.
\end{description}

The pattern analytic philosophers have of suspecting that if someone
isn't saying precisely something, they aren't saying anything at all, is
sometimes rather annoying. But the alternative, of saying allusive
things but not quite sticking to your own terms, isn't fun either.

\hypertarget{aside-on-mathematics}{%
\subsection{Aside on Mathematics}\label{aside-on-mathematics}}

Is mathematics objective in Williams's sense? As I read the text, he's
assuming that it must be, and that it's a constraint on a decent theory
of objectivity that it turns out to be objective. And he's right to be
scared of this, because it really isn't obvious on his theory that
mathematics is objective.

There is a big question here - do we believe mathematical claims because
they are true? This is hard to adjudicate because we normally understand
those `because' claims in terms of counterfactuals, and counterfactuals
about math being false are literally incoherent. So there is a worry
that mathematics won't be objective on Williams's view.

Williams suggests another way to be a realist (or objectivist) about
math - all the parts of it could in principle be believed together. But
I think this is true only on a particular view about what we're doing in
math, what we might call `if-then-ism.' On the face of it, it isn't true
that you can believe all of mathematics at once. Different set
theorists, for example, start with different axioms, and naturally prove
different things. You can't believe all of them. You can only have
Williams's attitude towards math if you think that these folks aren't
really proving theorems are true, they are just proving that certain
theorems follow from particular axioms. That's not an absurd theory of
mathematics - some people endorse it - but it should be controversial.
Like with the discussion of science, this reads like someone who never
actually engaged with real cutting edge technical work, and is just
going off what they learned in school.

\hypertarget{knowledge-involving-thick-concepts}{%
\subsection{Knowledge Involving Thick
Concepts}\label{knowledge-involving-thick-concepts}}

Williams thinks that at least some people in cultures unlike ours can
have knowledge involving thick concepts. Let's take a particular
instance of this, involving the character Laertes from \emph{Hamlet}.
Hamlet himself is exactly the kind of person who in questioning those
norms that hold society together undermines the confidence that Williams
is so keen on. But Laertes seems a pretty good 'un by Williams's lights
I reckon. Or at least not a bad 'un, and that'll be enough. So think
about (1).

\begin{enumerate}
\def\labelenumi{\arabic{enumi}.}
\tightlist
\item
  Laertes knows that it is honorable to kill Hamlet to avenge his
  father's death.
\end{enumerate}

Williams is interested in three challenges to this claim: one from
tracking, one from truth, and one from reflection.

The tracking concern is relatively easy to deal with. Laertes' beliefs
about honor do track something reasonably well. They aren't random. They
do accord with the cultural norms of honor. They aren't good beliefs.
They encourage revenge murder. As Ophelia points out, they lead to
hypocritical treatment of men and women - Laertes can have all the
affairs he wants off in Paris, but if she has an affair with Hamlet,
that's dishonorable. But still, Laertes isn't being random or capricious
here. We'll come back to whether conformity to culture is enough, but
Laertes does seem to be tracking.

The truth concern is a little trickier. If you understand honor as
meaning `good in virtue of X' for some descriptive property X, then you
will not think that 1 is true. It isn't good in virtue of anything for
Laertes to kill Hamlet. And here Williams simply wants to deny that's
how thick concepts work. They are action-guiding all right - Laertes
kills Hamlet because that's the honorable thing to do. But they are more
directly action-guiding. They don't tell you - oh this thing is
honorable, so it's good, so do it! Rather, they just say - oh this thing
is honorable, so do it! So Williams thinks at least this reason to think
that Laertes' belief is false fails. But what does he think the truth
conditions of Laertes' belief are? I'm not sure, and we'll come back to
this.

The reflection concern is in two parts. First, Laertes could not hold on
to this belief on critical reflection. Second, a belief is only
knowledge if it could survive this kind of reflective process. I want to
say more soon about reflection. Williams connects it to the Socrates,
and we could think of it as being the attempt to survive the process of
questions and answers that you see in the Socratic dialogues. But you
could also think of it in terms of what Hamlet does - spending hours
pacing around wondering about the nobility or whatever of various acts
you're contemplating. Anyway, Williams thinks that you can know things,
even if this knowledge wouldn't survive reflection. As he says,
reflection can destroy knowledge. So the fact that Laertes is not
reflective, and that he would have different beliefs if he were
reflective, doesn't threaten his knowledge.

But even though 1 is true, Williams wants to reject 2.

\begin{enumerate}
\def\labelenumi{\arabic{enumi}.}
\setcounter{enumi}{1}
\tightlist
\item
  It is an objective fact that it is honorable for Laertes to kill
  Hamlet to avenge his father's death.
\end{enumerate}

Why? Well it fails most of the conditions for objectivity mentioned
above. We can explain why so many people in the castle, and perhaps in
the broader culture, agree about honor without invoking honor. The best
explanation will be the social utility of concepts like honor. (Or at
least their utility to the powerful people.) So this won't be an
objective fact.

But there is this further question about what those of us who reject the
honor culture should say about what it is honorable for Laertes to do.
Williams's own answer is hard to parse. He gives this super weird
example involving schoolkids, and then hand waves about it being
something vaguely like that. But the example doesn't make a whole lot of
sense. I'm used to the idea that there are people who can mention but
not use a particular charged term. (That's a view about slurs, though
one that isn't as popular now as it was 10 years ago.) But what is going
on with a term that some of us are only allowed to mention or use in
indirect speech reports? Could we also use it in describing the beliefs
that the schoolkids have? I don't get it.

And this is the super-hard question. There are two obvious options
around here.

\begin{enumerate}
\def\labelenumi{\arabic{enumi}.}
\tightlist
\item
  Say that honor-talk is evaluative, and since we reject the
  evaluations, we don't go in for that kind of talk at all - even to the
  extent of attributing knowledge to Laertes. I think that's Foot's
  view.
\item
  Say that honor-talk is purely descriptive, though it maybe implicates
  or something that we endorse an evaluative picture. So we won't use it
  unless there is a way to cancel that implicature. That's Pekka's view
  more or less.
\end{enumerate}

Now Williams I think wants to slide between these two views. But I don't
see the space there. This is surely more a reflection of my lack of
insight than anything else. But I don't know what the positive picture
is supposed to be.

\hypertarget{two-notions-of-reflection}{%
\subsection{Two Notions of Reflection}\label{two-notions-of-reflection}}

Williams talks a lot in these chapters about reflection. But as I read
it, there are two different kinds of reflection, and they play different
theoretical roles.

\begin{itemize}
\tightlist
\item
  One kind is reflection by theorists who are looking to explain the
  uses of a thick term, and especially to explain convergence in its
  use.
\item
  Another is reflection by users of the term who are worried about
  whether they ought to be using the term.
\end{itemize}

The first kind of reflection plays a major role in chapter 8. It is
because our reflections on honor-talk go a certain way that honor-talk
is not objective in Williams's sense. The second kind of reflection
plays a major role in chapter 9. It's what Hamlet does, and it's what
leads to the terrible crisis of confidence that Williams is upset about
at the end of the chapter.

Now it is possible that these don't go the same way, especially on some
ways of solving the problems at the end of the last section. In
particular, some ways of solving those problems will end up with us
saying sure they converge on beliefs about honor because those beliefs
are true. Now what makes those beliefs true will not be remotely what
they think makes those beliefs true. It will be something about social
practices not something about true morality or whatever. But that's ok -
as David Chalmers points out, our beliefs about tables, chairs and beer
mugs aren't made true by what we think they are made true by.

The Hamlet-style reflection by users is a different story. (I'm maybe
cheating a little bit here. Hamlet worries about, e.g., whether suicide
is noble, not about whether nobility itself is good. But I think the way
he does it really does undermine the very notion of nobility.) That does
seem to require stepping back and using very different notions, in a way
that maybe (maybe) reflection on our color talk does not.

But I was surprised at how quickly Williams moves to thinking that we'll
have to use thin concepts to judge thick concepts. We could just use
other thick concepts. And in fact we often do. We often ask whether the
way we use certain terms or concepts is \textbf{just}. When Laertes sees
the error of his ways, and perhaps of the honor code he was working
under, he says that he is \emph{justly} slain, not \emph{rightly} slain.
More recently, Miranda Fricker (2007) talks about judging epistemic
concepts by the standards of epistemic \textbf{justice}. Maybe that's
wrong, we should not focus on epistemic justice and injustice but, as
Kristie Dotson suggests, epistemic \textbf{oppression}. There are lots
of thick concepts we have around for use in judging other thick
concepts.

Justice in general is a bit of a problem for Williams, and maybe we can
talk about that more in class. It doesn't really fit into his discussion
of either thin concepts or thick concepts, and this I suspect reveals a
weakness in the whole structure.

\hypertarget{relativism}{%
\subsection{Relativism}\label{relativism}}

This has gone on a bit, but I want to end with a discussion of one brief
argument in chapter 9 against a certain kind of relativism.

\begin{quote}
Consider once more the hypertraditional society, and suppose that it
does have some rules expressed in terms of something like ``right'' and
``wrong.'' When it is first exposed to another culture and invited to
reflect, it cannot suddenly discover that there is an implicit
relativisation hidden in its language. It is will always be, so it
speak, too early or too late for that. It is too early, when they have
never reflected or thought of an alternative to ``us.'' (A question from
Chapter 7 applies here: how could this have come into their language?)
\end{quote}

So this I think is badly mistaken, but for interesting reasons. I think
there is an implicit two step argument here.

\begin{enumerate}
\def\labelenumi{\arabic{enumi}.}
\tightlist
\item
  The locals are in control of at least the syntax of their language, of
  how many argument places their predicates have.
\item
  If relativism is true, then terms like ``right'' and ``wrong'' have
  one more argument place than people thought they did.
\item
  So, relativism is wrong.
\end{enumerate}

Premise 1 is fairly debatable - I think Williams is assuming some kind
of internalism about the language, at least at some level. And that's
controversial. You certainly don't want an individual level version of
that principle, as we can see by thinking about time zones. But set
worries about premise 1 aside, and focus on premise 2.

Think about the (very) ordinary sentence \emph{It's raining}. Call it R.
We know what it is for an utterance of R to be true. It has to be
raining

\begin{itemize}
\tightlist
\item
  In the place of the utterance;
\item
  At the time of the utterance;
\item
  In the world of the utterance.
\end{itemize}

When I utter R on Christmas Day in Alice Springs, it doesn't matter
whether it is or isn't raining somewhere else, or on some other day, or
in some other possible world. As long as it is raining at that place, at
that time, in that world, my utterance is true.

But this massively underdetermines what the content of R is. In general
we want to say that a sentence is true if its content corresponds to the
way things are. But when you have something like R, it is tricky to know
what to put into the content, and what to put into the way things are.
And in fact there is a choice to be made on each of the three bullet
points.

Most theorists, in fact everyone I know, thinks that the place goes into
the content. That is, an utterance of R expresses a content that has the
place of the utterance as a constituent, and the truth of that content
is measured against a `place-neutral' world. The alternative would be to
say that the content is just that rain is happening, and it is measured
against a world that looks a bit like a map with a ``You are here''
icon, and the utterance is true if it is raining at the icon. This is
coherent, but I don't think anyone thinks it. (In part because it yields
weird predictions about embeddings.)

Most theorists, but not quite everyone I know, thinks that the world
does not go into the content. An utterance of R does not express that it
is raining (in Alice Springs or wherever) \emph{in this world}. Rather,
which possible world we're in is part of the facts that the content is
measured against. We could go the other way around, and say that an
utterance of R has a world-specific content. And there is some evidence
for that approach. But mostly people think it's wrong.

And there is no consensus about how to handle time. \emph{Eternalism},
in this context, is the view that the content of R includes the time of
utterance, and that content is measured against a `timeless' world.
\emph{Temporalism} is that the content is just that it is raining (at a
place), and the world it is measured against has an icon or something
saying ``This is now.'' And what matters is that it is raining (at a
particular place) when that icon occurs. Among philosophers of language,
Eternalism is the majority view, but Temporalism has a number of
adherents, including some very prominent ones.

Why does this all matter? Because we can do the same trick about
cultural relativity. Even if sentences like ``It is honorable to kill
people who kill your father'' are culturally relative, we can ask
whether the relativity goes in the content, or in the world they are
measured against. And Williams's argument assumes, without any argument
at all, that the relativity goes in the content. And this is probably
the wrong form of relativism.

Well this isn't a philosophy of language seminar (really it isn't), so I
won't keep going. If you're interested (or even remotely curious) you
should read John MacFarlane's excellent book \emph{Assessment
Sensitivity}, which as well as going into great detail on this, will
also suggest ways in which the setup I've given here is somewhat
mistaken. But that's a debate for another, very different, class. For
now I'll just note that there are a lot more options for relativism that
are taken seriously now than there were when Williams was writing, and
some of the arguments here fall short because of this restricted
vantage.

\hypertarget{refs}{}
\begin{CSLReferences}{1}{0}
\leavevmode\hypertarget{ref-vanFraassen1980b}{}%
Fraassen, Bas van. 1980. \emph{The Scientific Image}. Oxford: Oxford
University Press.

\leavevmode\hypertarget{ref-Fricker2007}{}%
Fricker, Miranda. 2007. \emph{Epistemic Injustice}. Oxford: Oxford
University Press.

\leavevmode\hypertarget{ref-SEPAnalysisKnowledge}{}%
Ichikawa, Jonathan Jenkins, and Matthias Steup. 2013. {``The Analysis of
Knowledge.''} In \emph{The Stanford Encyclopaedia of Philosophy}, edited
by Edward N. Zalta, Fall 2013. Metaphysics Research Lab, Stanford
University.
\url{http://plato.stanford.edu/archives/fall2013/entries/knowledge-analysis/}.

\leavevmode\hypertarget{ref-Nozick1981}{}%
Nozick, Robert. 1981. \emph{Philosophical Explorations}. Cambridge, MA:
Harvard University Press.

\end{CSLReferences}

\end{document}
