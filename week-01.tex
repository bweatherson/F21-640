% Options for packages loaded elsewhere
\PassOptionsToPackage{unicode}{hyperref}
\PassOptionsToPackage{hyphens}{url}
%
\documentclass[
]{article}
\usepackage{amsmath,amssymb}
\usepackage{lmodern}
\usepackage{ifxetex,ifluatex}
\ifnum 0\ifxetex 1\fi\ifluatex 1\fi=0 % if pdftex
  \usepackage[T1]{fontenc}
  \usepackage[utf8]{inputenc}
  \usepackage{textcomp} % provide euro and other symbols
\else % if luatex or xetex
  \usepackage{unicode-math}
  \defaultfontfeatures{Scale=MatchLowercase}
  \defaultfontfeatures[\rmfamily]{Ligatures=TeX,Scale=1}
  \setmainfont[BoldFont = SF Pro Text Medium]{SF Pro Text Light}
  \setmathfont[]{Fira Math}
\fi
% Use upquote if available, for straight quotes in verbatim environments
\IfFileExists{upquote.sty}{\usepackage{upquote}}{}
\IfFileExists{microtype.sty}{% use microtype if available
  \usepackage[]{microtype}
  \UseMicrotypeSet[protrusion]{basicmath} % disable protrusion for tt fonts
}{}
\makeatletter
\@ifundefined{KOMAClassName}{% if non-KOMA class
  \IfFileExists{parskip.sty}{%
    \usepackage{parskip}
  }{% else
    \setlength{\parindent}{0pt}
    \setlength{\parskip}{6pt plus 2pt minus 1pt}}
}{% if KOMA class
  \KOMAoptions{parskip=half}}
\makeatother
\usepackage{xcolor}
\IfFileExists{xurl.sty}{\usepackage{xurl}}{} % add URL line breaks if available
\IfFileExists{bookmark.sty}{\usepackage{bookmark}}{\usepackage{hyperref}}
\hypersetup{
  pdftitle={PHIL 640: Week One - Introduction},
  pdfauthor={Brian Weatherson},
  hidelinks,
  pdfcreator={LaTeX via pandoc}}
\urlstyle{same} % disable monospaced font for URLs
\usepackage[margin=1.5in]{geometry}
\usepackage{graphicx}
\makeatletter
\def\maxwidth{\ifdim\Gin@nat@width>\linewidth\linewidth\else\Gin@nat@width\fi}
\def\maxheight{\ifdim\Gin@nat@height>\textheight\textheight\else\Gin@nat@height\fi}
\makeatother
% Scale images if necessary, so that they will not overflow the page
% margins by default, and it is still possible to overwrite the defaults
% using explicit options in \includegraphics[width, height, ...]{}
\setkeys{Gin}{width=\maxwidth,height=\maxheight,keepaspectratio}
% Set default figure placement to htbp
\makeatletter
\def\fps@figure{htbp}
\makeatother
\setlength{\emergencystretch}{3em} % prevent overfull lines
\providecommand{\tightlist}{%
  \setlength{\itemsep}{0pt}\setlength{\parskip}{0pt}}
\setcounter{secnumdepth}{-\maxdimen} % remove section numbering
\linespread{1.18}
\ifluatex
  \usepackage{selnolig}  % disable illegal ligatures
\fi
\newlength{\cslhangindent}
\setlength{\cslhangindent}{1.5em}
\newlength{\csllabelwidth}
\setlength{\csllabelwidth}{3em}
\newenvironment{CSLReferences}[2] % #1 hanging-ident, #2 entry spacing
 {% don't indent paragraphs
  \setlength{\parindent}{0pt}
  % turn on hanging indent if param 1 is 1
  \ifodd #1 \everypar{\setlength{\hangindent}{\cslhangindent}}\ignorespaces\fi
  % set entry spacing
  \ifnum #2 > 0
  \setlength{\parskip}{#2\baselineskip}
  \fi
 }%
 {}
\usepackage{calc}
\newcommand{\CSLBlock}[1]{#1\hfill\break}
\newcommand{\CSLLeftMargin}[1]{\parbox[t]{\csllabelwidth}{#1}}
\newcommand{\CSLRightInline}[1]{\parbox[t]{\linewidth - \csllabelwidth}{#1}\break}
\newcommand{\CSLIndent}[1]{\hspace{\cslhangindent}#1}

\title{PHIL 640: Week One - Introduction}
\author{Brian Weatherson}
\date{August 30, 2021}

\begin{document}
\maketitle

\hypertarget{introduction}{%
\subsection{Introduction}\label{introduction}}

Like many issues in philosophy, even saying what the debate about thick
concepts or terms is about requires taking sides on a number of disputed
philosophical questions. Rather than do that, I'll follow common
practice and start by just pointing out various examples. I'll list
terms here because it's easier to list them than to list concepts,
though our interest here is if anything more on the concepts than the
terms.

So start by considering the words in these two lists.

\begin{itemize}
\tightlist
\item
  Honest, Fair, Brave, Friendly, Generous, Tactful
\item
  Wicked, Cowardly, Rude, Nasty, Selfish, Cruel
\end{itemize}

We will mostly recognise the first as terms of approval, even as
distinctively moral kinds of approval, and the second as terms of
disapproval. But that kind of boo/hooray doesn't exhaust what is
typically communicated by the terms.

It is perfectly consistent to say that something was good but not
tactful. It may even be consistent to say that it was tactful but not
good; let's come back to that. What matters for now is that saying
something is tactful isn't just to express approval of it; it's to
approve of it in a particular respect. And if the thing doesn't have
that respect, then it isn't tactful.

To take a very topical example, it would be very strange to describe the
withdrawl of US troops from Afghanistan as tactful. If you disapprove of
the withdrawl, then sure that's easy to explain; you don't like using
positive terms to describe something you disapprove of. But even if you
do approve of it, you might still think that while the withdrawl was
good, it wasn't good in the way that tactful things are good. You might
think that the US actions were not tactful, and this is for broadly
speaking descriptive rather than evaluative reasons.

The terms on these two lists are related to something like moral or
ethical evaluation. But really for any kind of evaluation we can think
of terms that seem to have these evaluative and descriptive aspects. So
think of all of these broadly aesthetic terms.

\begin{itemize}
\tightlist
\item
  Beautiful, Sublime, Stylish, Elegant, Balanced, Delicate
\item
  Garish, Messy, Crude, Ugly
\end{itemize}

The first group seem more or less positive, the second more or less
negative. But they don't just have these evaluative senses; they seem to
have descriptive aspects to them. One can like, and approve of, a piece
of music without thinking it beautiful. (How much of the Velvet
Underground's first album is beautiful? How much does that matter for
how good the album is? Can something have a beautiful taste or smell? I
don't think so; a paper plane might look beautiful, but can it taste
beautiful?) And the other terms are even more contentful
descriptively.\footnote{If this was an aesthetics seminar I'd spend
  hours at this point asking if there are any terms in aesthetics that
  are as `thin' as `good' and `right.' But let's leave that for another
  day.}

We can do the same thing for evaluations to do with rationality, whether
it is epistemic or practical.

\begin{itemize}
\tightlist
\item
  Shrewd, Open-Minded, Humble, Insightful, Perceptive
\item
  Gullible, Stubborn, Shallow, Lazy
\end{itemize}

We're going to cover epistemology a lot more in this course; let's leave
it there for now.

So far I've been dividing up the terms into the good 'uns and the wrong
'uns, but one thing that is super important is that there are plenty of
terms that are nowhere near this easy to classify. The main book for
this course, Pekka Vayrynen's \emph{The Lewd, The Rude and the Nasty}
(Väyrynen 2013), makes these terms the centerpiece of the main argument.
So let's list a few contentious ones already.

\begin{itemize}
\tightlist
\item
  Lewd, Chaste, Blasphemous, Sinful
\end{itemize}

The idea that Vayrynen wants to build up is that these terms encode a
moral system that we might well want to reject. Oddly, although Vayrynen
wants to disagree with Bernard Williams (1985) about approximately
everything to do with these thick terms, the possibility of such terms
is also central to Williams's story. That's because Williams thinks that
these terms are most at home in particular moral systems that need not
be universally shared. One of his aims on getting us to focus on thick
terms is to make moral relativism more appealing. Different cultures use
different thick terms, but they are still using moral theories when they
do so. But these are different theories, so of course the terms from
other systems will strike us as defective in some way or other.

There are three more contentious terms that I want to spend a little
time on before we move on.

One is really a class of terms: slurs. These seem to also have
descriptive and evaluative components. If you call someone a bogan or a
chav you're probably insulting them.\footnote{If you don't know either
  of these terms,
  \href{https://www.urbandictionary.com/define.php?term=bogan}{Urban
  Dictionary} might help - though probably the `definitions' there will
  be just as unintelligible as the terms. Anyway, these are both broadly
  social slurs aimed at a particular kind of less wealthy person.} But
you're also making a descriptive claim, one that could be wrong. Donald
Trump, for instance, is neither a bogan nor a chav, even if much of his
sales pitch is targeted at people who might well fall into one or both
of these categories. And this seems to be the general case with slurs;
they have both descriptive and evaluative elements. But for most slurs,
maybe including `bogan' and `chav,' the evaluative aspect is based in a
bad moral theory. This is most obvious for the most obvious racial
slurs, but it's kind of true across the board. This isn't a seminar on
slurs, but slurs will be often in the background.

The second pair, which we will spend a lot of time on, is `humble' and
`modest.' I'm not really going to separate these two, though maybe if
it's interesting enough we can. What I will spend a lot of time on is
whether these are positive terms at all, and if so how they are. There's
that old line, sometimes attributed to Winston Churchill, about a modest
man with much to be modest about, that gets at why you might worry about
these being positive.\footnote{Though since the target of the quip is
  usually said to be the person who was in fact the greatest ever
  British PM, maybe this example doesn't show very much.} And we'll see
in the philosophical theories of them, it's very hard to come up with
stories about humility and modesty that make them unambiguously
positive.

The third kind of case I want to think about, and really just to flag,
is that with a strange enough moral theory, almost any term can be
positive. Think about `selfish' as used by a devotee of Ayn Rand. This
is going to be more grist for the view that Vayrynen pushes, that these
terms are not actually evaluative at all, just descriptive. We'll spend
enough time on that in later weeks; I'm just flagging it now.

As well as the contentious terms, we should also think about the terms
that are in a sense primarily descriptive, but which one could object to
on evaluative grounds. Here are some examples, which will be relevant as
we go along.

\begin{itemize}
\tightlist
\item
  Murderer, Thief, Liar
\item
  Footballer, Artist, Philosopher
\item
  Joke, Artwork, Song
\end{itemize}

The first three rely not just on something happening, e.g., some object
being acquired, but on an evaluation of that happening, e.g., that the
acquisition was of something a different person had a claim to. The
second three rely on at least a minimal standard of quality being met.
Whether a teenager is a footballer depends not just on whether they kick
a ball around, but on how well they do it. There are bad footballers,
but if you're bad enough, you're no longer a footballer. And the same
goes for the last category. A bad enough joke isn't actually a joke.

If you think of thick/thin as a spectrum, and I think it's more or less
a consensus now that you should, then these terms are at the opposite
end from terms like `wicked,' `intelligent,' `beautiful,' and other
terms that don't tell you very much about the descriptive attributes of
the thing under discussion. But they still have many things in common
with other thick terms, and it's worth remembering their existence when
we consider theories of thick terms.

One last introductory point that will be very relevant for the second
half of the course. Most of these terms can be applied either to token
actions/thoughts/intentions, or to agents as temporally extended beings.
But, at least intuitively, they seem to apply in the first instance to
agents, not to actions. That's to say, at least intuitively the
following theories seem like they are on the right track to me.

\begin{itemize}
\tightlist
\item
  For someone to be a good person just is for their actions (or perhaps
  intentions) to be good.
\item
  For something to be a generous action (or intention) just is for it to
  be the kind of thing a generous person would do.
\end{itemize}

Focussing on these terms, or concepts, draws our moral attention to
people, in all their messy complexity, and away from individual actions
or thoughts. And maybe that's a very good thing. Whether or not it is,
it's quite a momentous thing, and it's something we should pay attention
do before jumping ahead.

\hypertarget{puzzles-about-definition}{%
\subsection{Puzzles about Definition}\label{puzzles-about-definition}}

So far I've I ostended over 30 terms I've called thick terms, as well as
noting that the class includes slurs, so you can probably add that many
more again to the class off the top of your head.\footnote{How many more
  might depend on your cultural upbringing; the variety of slurs in
  circulation seems to vary a lot from place to place.} It would be good
at this point to have a neat definition of our subject matter, to say
what it is for something to be a thick term.

Unfortunately, that isn't going to happen. Any proposed definition will
involve way too much theory. People who disagree with the theory can
more-or-less agree on the existence and rough extension of the
distinction we're setting out. So the definition of the distinction
can't be in these theoretically contentious terms.\footnote{Is this an
  open question argument? Maybe! Foreshadowing\ldots{}}

For instance, you might borrow the venerable fact/value distinction and
say that a thick term is one that has both factual and evaluative
components. To say that Gen is a thief is to say that he brings things
into his proximity, factual, and that he shouldn't do this, evaluative.
But this is wrong for at least four separate reasons.

\begin{enumerate}
\def\labelenumi{\arabic{enumi}.}
\tightlist
\item
  If you're a moral realist, you'll think it is a fact that it is wrong
  for Gen to bring these things that belong to other people into his
  proximity.
\item
  This way of putting it assumes that we understand what it is for terms
  to be evaluative, and indeed that terms are evaluative, and both of
  these are contentious. (Note that Kirchin (2018) thinks that a big
  part of what's wrong with the literature is that we've misunderstood
  how terms are evaluative.)
\item
  This way of putting it assumes something like Separability; the idea
  that thick terms have two different parts, one evaluative and one
  `factual.' And that's not just contentious; it's mostly taken to be
  false.
\item
  And the whole setup assumes a fact/value distinction - and the reason
  these terms were first made salient in philosophy was to undermine the
  very existence of that distinction. Much more on this to come.
\end{enumerate}

Williams (1985) goes a different way. He says that the distinctive thing
about these terms is that they are both `world-guided' and
`action-guiding.' I take it he's picking up here on something like the
idea of mental states having a `direction of fit' from Anscombe
(1957).\footnote{See Humberstone (1992) for much more background on
  this.}. We want both our beliefs and our desires to have true
contents. But what differs between beliefs and desires, at least in
paradigm cases, is how we react if they don't. If our beliefs have false
contents, we change the beliefs. If our desires have false contents, we
change the world. The world guides our beliefs, and our desires guide
our actions. Or at least that's the simple case; our attention will be
on the cases that are less simple. To hold that Gen is a thief, on this
picture, is to be in a state that should be guided by the world - it
should be a reaction to Gen bringing certain kinds of things into his
proximity - and should guide action - e.g., by punishing him. (Or, I
guess more likely, using better locks when he's around.)

But this won't work either, again for several reasons.

\begin{enumerate}
\def\labelenumi{\arabic{enumi}.}
\tightlist
\item
  Some moral realists, most notably I guess Murdoch (1970), insist that
  moral attitudes are world-guided.
\item
  Some philosophers, including me, deny that moral attitudes are
  (properly) action-guiding.
\end{enumerate}

I think we're best off just sticking with the lists.

\hypertarget{descriptive-equivalence}{%
\subsection{Descriptive Equivalence}\label{descriptive-equivalence}}

I want to flag a question for now because we're going to come back to it
a lot in what follows. Let's say you've got one of these
sorta-descriptive/sorta-evaluative terms. And for some reason you want
to factor out the descriptive part. Maybe you don't actually share the
views that make the evaluative part make sense. Or maybe you want to
explain the term to someone who doesn't. To make things concrete, maybe
you're trying to explain what a thief is to someone who really doesn't
get property norms, and can't figure out how they work.

It's actually kind of hard. Probably the best way to do it would be to
try again to explain the moral theory of property - not as something
they necessarily should endorse but as something they should understand
- and then say that to be a thief is to violate this theory in certain
ways. And maybe that could work, though I think it's going to be tough.

This wasn't much of an argument - we'll see more thorough arguments in
weeks to come - but it suggests the following conclusion.

\begin{itemize}
\tightlist
\item
  There is no term that has the same descriptive meaning as `thief,' but
  without the moral connotations.
\end{itemize}

In the language of thick concepts, there is no \textbf{descriptive
equivalent} to `thief.' The parallel claim in the slurs literature would
be that a particular slur, like `bogan,' has no neutral counterparts.
And I suspect both of them are right.

What would follow if this claim is right? We'll spend a lot of time on
this too. But one thing that you might think follows is that it's bad
news for views that say thick terms can be factorised into a descriptive
and an evaluative part. And that in turn would be bad news for theories
that require the possibility of such factorisations. It's time to turn
to such theories.

\hypertarget{self-indulgent-interlude}{%
\subsection{Self-Indulgent Interlude}\label{self-indulgent-interlude}}

The reason I'm interested in this stuff is that I'm in print defending
what sure looks like one of those theories. Now I want to stress that
theories like mine are \textbf{not} why people have been typically
interested in thick concepts. But they are relevant to me, and it's my
seminar, so let's start.

One of the core theses of \emph{Normative Externalism} (Weatherson 2019)
was that it is not good to be motivated by doing good as such. So I want
to contrast these two cases.

\begin{itemize}
\tightlist
\item
  There is a drowning child in the pond. Jill does not want the child to
  drown, so she jumps in and rescues the child.
\item
  There is a drowning child in the pond. Jack doesn't care about whether
  or not children drown, at least under that description. But he does
  care a lot about doing the right thing, whatever it turns out to be,
  and he believes (correctly) that right now doing the right thing would
  be jumping in to rescue the child. So he jumps in and rescues the
  child.
\end{itemize}

I'm committed to the view that what Jill did is much better, morally
speaking, than what Jack did. Officially in the book the view is that
being motivated by the desire to do the right thing, whatever it is, has
no positive moral value. Actually I think something stronger than this -
that it's a vice.\footnote{We maybe even have a word for it in English:
  `self-righteousness.' But whether that's exactly the right word for
  this character trait is less interesting to me than whether the trait
  is a vice}

There is a common argument for the kind of view I have: the Nazi
argument. The Nazis were horrible in every respect, yet at least some of
them were motivated by a desire to do the right thing - while of course
having horrible beliefs about what was right. So the fact that the Nazis
were trying to do the right thing - under that very description (or at
least its German translation) - isn't a virtue of theirs. This argument
is super sloppy, and I don't really feel like trying to tidy it up,
because I don't think using Nazis in this kind of debate is helpful.
There are messy historical questions about just what kind of motivations
different Nazis actually had. And even if we sort those out, how to map
Nazi attitudes onto regular categories from moral psychology is a hard
and unpleasant question.\footnote{The evidence in Ohler (2016) that they
  were taking all the drugs does not make this task any easier.}

A better argument comes from thinking about terrorist groups, especially
groups that have self-consciously moral aims. And within those,
especially groups whose ends are ones that we might find basically
congenial, however much we disagree with them about means. So in the
book I spend a bit of time talking about Jacobins, mostly drawing on the
discussion in Palmer (1941). But I suspect it would have been better to
focus as well, or perhaps instead, on IRA terrorists, as discussed in
Keefe (2019). I just don't think you should look at people murder single
parents because of flimsy evidence that they were providing trifling
help to the government, and say that ``Well, at least they thought they
were doing the right thing.'' Instead, I think you should see their
overarching desire to do the right thing as having removed a very useful
guardrail that most of us have on our actions - that little voice
Socrates talks about that warns us against doing something so abhorrent.

So what I'd really like to say is this. A good action just is one that
is driven by good intentions. And good intentions are intentions that
are driven by desires to promote descriptively individuated states that
are in fact good, while also desiring to avoid (or perhaps prevent)
descriptively individuated states that are in fact bad. It's to be like
Jill - to jump in because a child is drowning.

Now if there was a clear descriptive/evaluative distinction, that would
all be fine. I mean, it would still be controversial, but it would at
least make sense. But what do we say about someone who\ldots{}

\begin{itemize}
\tightlist
\item
  did X because it was the only fair thing to do?
\item
  didn't do X because it would be stealing?
\item
  did X because it was the bravest thing to do?
\item
  didn't do X because it would be lazy to do it?
\end{itemize}

I have views about each of these questions. But I don't have anything
remotely like a theory that produces these verdicts as theorems. So this
is why I'm interested in the topic. But it's not the standard reason.
That has to do with anti-realist ethical views.

\hypertarget{non-cognitivism-and-expressivism}{%
\subsection{Non-Cognitivism and
Expressivism}\label{non-cognitivism-and-expressivism}}

In mid-century England, the dominant meta-ethical view was a kind of
anti-realism. This view had two components. Both of them are, I think,
best understood comparatively.

One component of the view concerns mental content. We have a rough idea
of what it is to believe a descriptive proposition about the world, like
that the cat is on the mat.\footnote{Actually, you probably don't want
  to think too hard about what kind of attitude those folks thought was
  involved in believing the cat is on the mat. But it's a world-directed
  notion, in some sense of `world' and `directed'} To hold that murder
is wrong is not, they think, anything like that. Maybe it's to
disapprove of murder. Maybe it's to desire not to murder, or that anyone
else murders. Maybe it's also a desire to punish in some ways those that
do murder. What it isn't is a belief that this action type, murder, has
some moral property, being wrong. And that's because there is no such
property.

Given this view about mental content, a view about linguistic content is
just about forced. To say that murder is wrong is not to describe the
world in any way. Again, don't think too hard about what makes something
a description - just that it's what we do when we say that the cat is on
the mat. We're not doing that when we say murder is wrong. What are we
doing? Well, in the crudest version of the view, more or less the one
defended by Ayer (1936), we're saying \emph{Boo murder!}. A moral
philosophy talk, properly interpreted, sounds like the crowd at some
underperforming football club.\footnote{Sorry I guess to fans of
  Arsenal/Newcastle/Schalke or whoever is triggered by this analogy.}
The view on mental content is \textbf{non-cognitivism}, the view that
moral attitudes are not cognitive states. The view on linguistic content
is \textbf{expressivism}, that the role of moral terms is to express
these (non-cognitive) states.

These are normally taken to be views in meta-ethics; they aren't moral
theories, but views about what moral theory is. And towards the end of
the 20th century, the focus of the debate around it turned on technical
questions like how moral terms behaved in antecedents of conditionals.
Since \emph{If Boo murder!, \ldots{}} isn't even grammatical, it's hard
to explain what \emph{If murder is wrong,\ldots{}} means.

But in the 1950s, that wasn't the concern. Rather, the concern was that
this meta-ethical theory led to bad moral theory. Or, almost as bad,
that it led to not actually doing moral theory. Since we shouldn't
really spend too much time theorising about what we cheer or boo, we
shouldn't have moral theory be a central part of philosophy. Actually
put that bluntly that doesn't sound like a great argument. But as a
matter of fact normative ethics had fallen in importance, and these
kinds of anti-realist views, non-cognitivism and expressivism, had taken
hold, and maybe there was a causal connection there.

Now the first big entry of of thick concepts into the philosophical
debate comes as a response to this kind of anti-realism. The term `thick
concept' isn't used, but the idea is right there. And it's not attacking
anything about the details of one or other account of what non-cognitive
state a moral attitude is, or just how moral terms behave in
antecedents. It's really a straight up attack on the very idea that
there is a boundary between the moral and the descriptive, a boundary
whose existence the non-cognitivist needs to presuppose. And this is
what you see especially in the work of Philippa Foot (1958) and Iris
Murdoch (1956). We'll spend much more time on these next week, so I'm
just going to flag them here.

But why did people have these kinds of anti-realist views in the first
place? The simple answer is that it was a response to the arguments that
G. E. Moore (1903) made, and we'll spend the rest of this week looking
at those arguments.

\hypertarget{self-indulgent-interlude-2-history-of-analytic}{%
\subsection{Self-Indulgent Interlude 2: History of
Analytic}\label{self-indulgent-interlude-2-history-of-analytic}}

There is a standard story about the history of analytic philosophy. In
this history, logic and philosophy of science play a particularly
central role. A lot of the tools that are central to today's philosophy
of language are developed over the story, but the lead characters don't
care that much about natural language, so the relation between this work
and philosophy of language is complicated. But logic and philosophy of
science are central to the story.

In the standard story, there are three big figures towering over the
field: Frege, Russell, Wittgenstein. Note that two of these wrote in
German, and one of them was in a math department not philosophy. Those
features are not atypical. Most of the other important figures wrote in
German, and several of them would have identified as mathematicians
rather than philosophers. So think of the philosophical importance of
figures like Hilbert, Gödel, Tarski, Schlick, Carnap, Popper and so on,
with Cantor as an important influence if you reach further back.

But while these figures are important, I think this leaves a huge amount
out of the story. In particular, it leaves out value theory. In much of
the 20th century, value theory had a split role in the university, as
you see in the title of Liz Anderson's first book: \emph{Value in Ethics
and Economics} (Anderson 1995). Very roughly speaking, the more
quantitative part of value theory was done in economics, the more
qualitative part in philosophy. That was never wholly true, and with the
recent rise of formal ethics is less true than before, but it wasn't
wholly false for many decades.

In the 19th century though, especially in England and Scotland, there
really wasn't a distinction here. The people we now think of as the most
important figures in economic thought all had very thorough training in
philosophy. And the best philosophers read, and often contributed to,
economic thought.

I think the history of analytic philosophy should include, and arguably
start with, some of these philosopher-economists. The most important
figures here are probably Jevons (to my mind the first in the tradition
we call analytic philosophy), Edgeworth and (though he's a more
complicated case) Marshall.

Why am I including this here? Because the peak of this tradition came
with two great philosopher-economists who interacted with Moore, namely
Keynes and Ramsey. Keynes's doctoral work was a theory of probability
that was designed to respond to some challenges Moore raised, and
Ramsey's theory of probability - which is probably still the dominant
version in philosophy today - is a response to some issues in Keynes's.
There is this rich interaction between philosophy and economics that was
lost after World War Two, but which we should understand if we want to
know the history of our discipline.

We also have this amazing document of what it was like to do philosophy
at the time Moore was writing \emph{Principia Ethica}, Keynes's
posthumously published memoir ``My Early Beliefs'' (Keynes 1938). I'm
not sure everyone would call it a philosophy paper, but it's my favorite
ever philosophy paper, at least to read. I learn a bit more about the
time period every time I read it, so I don't want to leave you with the
impression that there is only one thing to learn about it. But here's
what I want to stress for today.

In the middle of Keynes's paper there is this long list of questions
about, as he puts it, mensuration. Some of them feel like how many
angels on the head of a pin. But some of them are just what we spend
time talking about today. Just to pull out one example, though
completely not at random.

\begin{quote}
In valuing the consequences did one assess them at their actual value as
it turned out eventually to be, or their probable value at the time? If
at their probable value, much evidence as to possible consequences was
it one's duty to collect before applying the calculus?
\end{quote}

Now what I want to stress is how unlike this is to everything else
that's happening in philosophy at the time. Pick up any philosophy
journal and you will find lots of articles about the nature of the
Ideal, and very few about duties to collect evidence before
action.\footnote{I think that, for example, Bosanquet's ``Hedonism Among
  Idealists'' (Bosanquet 1903) is much more representative of the
  period.} But a lot of what Moore discussed, along with his very
devoted followers, seems continuous with what we do today.

\hypertarget{the-open-question-argument}{%
\subsection{The Open Question
Argument}\label{the-open-question-argument}}

Moore has been so influential on meta-ethics over the last 120 years
that it is easy to place him using familiar language. That's because the
language developed in no small part in response to his work.\footnote{I'm
  drawing a lot on Hurka (2021), especially in the first few paragraphs.}

Moore is a moral realist. He thinks that moral properties exist, that
some things have them and that others don't. At least in \emph{Principia
Ethica}, he has an extremely anti-relativist form of realism, though
that gets at least a bit qualified in later works.

He is, in the language that Vayrynen uses, a thin centralist. He thinks
the key notions in ethics are things like goodness, rightness and what
one ought, all things considered, do.

He is a consequentialist. He thinks that what makes something the right
action is that it produces the best consequences. In Rawls's
terminology, he thinks that the good is prior to the right.

And he is an agent-neutral consequentialist, again at least in
\emph{Principia Ethica}. He thinks that what is absolutely good is prior
to what is good for me or for you, if we can even make sense of those
notions.

But he is not in any recognisable sense a utilitarian. He is a value
pluralist; he doesn't think that there is just one thing we should try
to maximise. And that one thing is not purely mental. He thinks,
anticipating some environmental ethics, that it would be better for the
world to be beautiful rather than ugly, even if no one was around to
appreciate it. And, famously, he believes in the principle of
\textbf{organic unity}, or what we might call value holism. The value of
a whole is in no sense the sum of the value of the parts.\footnote{This
  is implicit in Jevons's work, but it's Moore who really makes it
  central to philosophical discussions of value.} I don't know precisely
how we want to use the term `utilitarian,' but I think it probably
should be reserved for people who believe at least some of the things
that Moore is denying here.

And now we get to the most contentious term of all: Moore is a
non-naturalist. Just what this means is not easy to say. It does not
mean that he thinks goodness is some kind of supernatural property. The
way Moore uses words, divine command theories are naturalist. This is
one bit of terminology that has not caught on. The simplest thing to
say, in modern lingo, is that Moore is an anti-reductionist.

It's this anti-reductionism that is meant to be supported by the
infamous \textbf{open question argument}. Here's how a modern
presentation of the argument might go. We'll come back to how much Moore
says anything quite like this. Consider any proposed reduction of
goodness to some descriptive property D. According to the reduction,
goodness is D. But we can sensibly say, ``That is D, but is it good?''
Its goodness might remain an open question for us even after we concede
it is D. So goodness is not D. And since D was arbitrary, it follows
that goodness isn't anything else.

As it stands, there is a simple response to this argument. Who is the
`we' who can sensibly ask the open question? It can't be those who know
that goodness is D - the question won't be open for them. Moore thinks
there are no such people, but he can't take that as a premise; that
would make the argument obviously circular. A natural hypothesis is that
it is competent users of the language. But now we have the familiar
point that competent speakers of the language need not be on top of all
reductions. We have a successful reduction of water, it is
H\textsubscript{2}O. But for someone who doesn't know this bit of
chemistry, the question ``Does this water contain oxygen?'' might be an
open question. So if you restrict things to people who know that
goodness is D, the question is not open, and if you don't, the openness
of the question doesn't prove anything.

I think this is a pretty conclusive response to what I was calling the
open question argument. But there are at least three reasons to think
that at this point we've misinterpreted Moore. First, nothing that has
been said so far distinguishes `good' from `horse.' In both the argument
and reply, you could replace `good' with `horse' and everything else
would go through smoothly. Yet whatever else Moore is doing, he is
trying to show a way in which `good' is different from `horse.' Second,
this makes the argument sound like it really is about meanings of words.
And again, Moore is very clear that he doesn't care about meanings in
the ordinary sense. Think, for example, about the parts where he is
mocking someone who thinks that the fact that something falls in the
extension of a particular English word is a reason to do it. He does
talk about definitions, and we'll have to come back to that, but he
isn't talking about word meanings. Third, there seem to be two arguments
in §13, which looks like the key section, and neither of them seem to be
the modern presentation of the argument - though both of them eventually
get close to it.

So what is Moore trying to do, if he's not providing an argument about
the meaning of `good?' He is asking what the \textbf{metaphysical
definition} of goodness is. This idea of there being metaphysical
definitions somewhat dropped out of philosophy for a while - it really
is the kind of metaphysics that the positivists wanted to be rid of -
but you can see it in some contemporary philosophy. I think what Moore
is up to here is continuous with the discussions of metaphysical
semantics in \emph{Writing the Book of the World} (Sider 2011). Moore
thinks that the language of the book will include a word much like
`good,' but it won't include a word like `horse.' If you want to really
say how things are, you shouldn't talk about horses; you should talk
about horse parts and their arrangement. But you should talk about
goodness.

What then is the argument for that conclusion? It seems on the face of
it Moore gives two arguments, one in §13(1), the other in §13(2). And
neither I think is the standard open question argument. The argument in
§13(1) goes like this.\footnote{And note that Moore's particular example
  of D involves second-order desires, just like the theory in Lewis
  (1989). It's really so much more contemporary sounding than so much
  else that was going on at the time.} Let D be an arbitrary proposed
definition of goodness; we'll show it fails.

\begin{enumerate}
\def\labelenumi{\arabic{enumi}.}
\tightlist
\item
  It makes sense to ask whether it is D that X is D. E.g., it makes
  sense to ask whether it is desirable to desire that we desire to
  desire beauty.
\item
  It does not make sense to ask whether it is good that X is good.
\item
  So goodness is not D.
\end{enumerate}

Or perhaps the argument is this - I really can't tell quite what's going
on in premise 2.

\begin{enumerate}
\def\labelenumi{\arabic{enumi}.}
\tightlist
\item
  It makes sense to ask whether it is D that X is D. E.g., it makes
  sense to ask whether it is desirable to desire that we desire to
  desire beauty.
\item
  Anyone who understands the question of whether it is D that X is D
  will understand that it is not the same question as whether it is good
  that X is D.
\item
  So goodness is not D.
\end{enumerate}

The point is not we might be competent users of a term and not recognise
correct metaphysical reductions. That's an obvious consequence of
thinking about `horse' for a minute. Rather, it's that fully informed
users of the terms will recognise the differences between these
questions.

Why believe premise 2 in either case? Here's where I think Moore gets
closest to the standard presentation of the argument. What he really
needs is that fully informed speakers of the language of the book of the
world will see that there is a difference between \emph{It is D that X
is D} and \emph{It is good that X is good} / \emph{It is good that X is
D}. Now his opponents think they will not. Who's right? Well the claims
sure seem different to us pretty-well-informed speakers, and maybe
that's evidence that they seem different to fully informed speakers. I'm
not sure. In any case, it's already a bit interesting that the argument
is about these claims where D appears twice; I don't think the standard
presentation does that.

The discussion in §13(2) gets closer to the standard presentation. But
it's very odd. I can't really tell who the opponent is supposed to be
here, and I suspect that's about to become important. He says in the
first sentence that the opponent is the person who says that `good' is
meaningless. I think this is the view that `good' doesn't have a
counterpart in the book of the world not because like `horse' it can be
reduced, but because like `phlogiston' it wasn't needed. But I'm super
not sure - that's the person he says he is targeting - but this doesn't
make best sense of what follows.

So what does follow. Well, let D1, D2, D3, etc be different proposed
reductions of goodness that people might have tried back in the previous
argument. Then we get the following argument.

\begin{enumerate}
\def\labelenumi{\arabic{enumi}.}
\tightlist
\item
  For each Di, it makes sense to ask whether Di is good.
\item
  Each of these questions has something in common.
\item
  If goodness had no meaning, these questions would not make sense, or
  at least wouldn't have anything in common.
\item
  So goodness has a meaning.
\end{enumerate}

Then Moore puts the two parts together. By the second argument, goodness
has to appear in the book of the world either directly, as part of the
basic language, or indirectly, as something else that gets reduced to
it. But by the first argument it can't appear indirectly. So it must be
a fundamental feature of the world. And that's what the non-naturalist
realist wants.

Now one giant caveat needs to be put on the board here. In both §13(1)
and §13(2), the discussion around the argument gets very close to what
I'm calling the standard presentation of the open question argument. But
ultimately I don't think that's the argument. I think the argument is
about the non-identity of second-order claims involving D, and the
intelligibility of questions about whether D is good. But on the latter
part what matters is not the possibility that such a claim is false, as
in the standard presentation, but that such a claim is true.

So that's Moore. Why is he relevant to our story? Well a lot of
philosophers thought that a lot of the arguments he made were
successful, but that his conclusion - non-naturalist realism - was
unacceptable. So they wanted a way out. And non-cognitivism was
advertised as being a way out of the argument.

If \emph{D is good} just means \emph{Hooray for D!}, then you can accept
a lot of Moore's premises without accepting his conclusions. Asking
whether it is good that X is D is different to asking whether it is D
that X is D. The latter is asking a factual question about the world;
the former is asking whether to cheer for a different fact. And asking
whether D1, D2 etc are all good does make sense, and the questions do
have something in common. They are all questions about what one cheers
for - that's the common factor. And of course this approach handles the
standard presentation of the open question argument - which probably
does show up in something like this form in Ayer (1936) - just as
smoothly. Of course it makes sense to ask \emph{X is D, but do I cheer
it?} for any D.

So that's where our story starts. Moore convinced people that a certain
kind of reductionist naturalism was a non-starter. Wanting to hold on to
ethics, but not at the price of non-naturalism, many philosophers
responded by adopting non-cognitivism. But this relied on a heavy-duty
fact/value distinction, and seemed to undermine the philosophical
significance of moral theorising. And the first significant push back
against non-cognitivism came from people who wanted to complicate the
fact/value distinction, with thick concepts as their primary weapon for
muddying the waters.

\hypertarget{for-next-time}{%
\subsection{For Next Time}\label{for-next-time}}

There is no class next week for Labor Day. I'm not going to write
anything this long for any upcoming class - I didn't mean this to be
this long. But we'll cover three things.

\begin{enumerate}
\def\labelenumi{\arabic{enumi}.}
\tightlist
\item
  Arthur Prior's objections to Hume's ``No ought from an is'' principle.
\item
  Philippa Foot's discussion in ``Moral Arguments,'' especially the
  discussion of rudeness.
\item
  If time, Williams's introduction of the term `thick concepts' in
  Chapter 7 of \emph{Ethics and the Limits of Philosophy}.
\end{enumerate}

Of these, the priority is point 2; I plan on having us go through that
paper pretty carefully. But hopefully we'll have enough time to say
something about Prior and Williams as well.

\hypertarget{references}{%
\subsection*{References}\label{references}}
\addcontentsline{toc}{subsection}{References}

\hypertarget{refs}{}
\begin{CSLReferences}{1}{0}
\leavevmode\hypertarget{ref-Anderson1995}{}%
Anderson, Elizabeth. 1995. \emph{Value in Ethics and Economics}.
Cambridge, MA: Harvard University Press.

\leavevmode\hypertarget{ref-Anscombe1957}{}%
Anscombe, Elizabeth. 1957. \emph{Intention}. Oxford: Basil Blackwell.

\leavevmode\hypertarget{ref-Ayer1936}{}%
Ayer, Alfred. 1936. \emph{Language, Truth and Logic.} London: Gollantz.

\leavevmode\hypertarget{ref-Bosanquet1903}{}%
Bosanquet, Bernard. 1903. {``Hedonism Among Idealists (i.).''}
\emph{Mind} 12 (46): 202--24.
\url{https://doi.org/10.1093/mind/XII.2.202}.

\leavevmode\hypertarget{ref-Foot1958}{}%
Foot, Phillipa. 1958. {``Moral Arguments.''} \emph{Mind} 67 (268):
502--13. \url{https://doi.org/10.1093/mind/LXVII.268.502}.

\leavevmode\hypertarget{ref-Humberstone1992}{}%
Humberstone, I. L. 1992. {``Direction of Fit.''} \emph{Mind} 101 (401):
59--83. \url{https://doi.org/10.1093/mind/101.401.59}.

\leavevmode\hypertarget{ref-Hurka2021}{}%
Hurka, Thomas. 2021. {``{Moore's Moral Philosophy}.''} In \emph{The
{Stanford} Encyclopedia of Philosophy}, edited by Edward N. Zalta,
{S}ummer 2021.
\url{https://plato.stanford.edu/archives/sum2021/entries/moore-moral/};
Metaphysics Research Lab, Stanford University.

\leavevmode\hypertarget{ref-Keefe2018}{}%
Keefe, Patrick Radden. 2019. \emph{Say Nothing: A True Story of Murder
and Memory in Northern Ireland}. New York: Doubleday.

\leavevmode\hypertarget{ref-KeynesMEB}{}%
Keynes, John Maynard. 1938. {``My Early Beliefs.''} In \emph{The
Collected Writings of John Maynard Keynes}, X:433--51. London:
Macmillan.

\leavevmode\hypertarget{ref-Kirchin2017}{}%
Kirchin, Simon. 2018. \emph{Thick Evaluation}. Oxford: Oxford University
Press.

\leavevmode\hypertarget{ref-Lewis1989b}{}%
Lewis, David. 1989. {``Dispositional Theories of Value.''}
\emph{Aristotelian Society Supplementary Volume} 63 (1): 113--37.
\url{https://doi.org/10.1093/aristoteliansupp/63.1.89}.

\leavevmode\hypertarget{ref-Moore1903}{}%
Moore, G. E. 1903. \emph{Principia Ethica}. Cambridge: Cambridge
University Press.

\leavevmode\hypertarget{ref-Murdoch1956}{}%
Murdoch, Iris. 1956. {``Vision and Choice in Morality.''}
\emph{Proceedings of the Aristotelian Society} 30 (1): 32--58.
\url{https://doi.org/10.1093/aristoteliansupp/30.1.14}.

\leavevmode\hypertarget{ref-Murdoch1970}{}%
---------. 1970. \emph{The Sovereignty of Good}. London: Routledge;
Kegan Paul.

\leavevmode\hypertarget{ref-Ohler2016}{}%
Ohler, Norman. 2016. \emph{Blitzed: Drugs in the Third Reich}. New York:
Mariner Books.

\leavevmode\hypertarget{ref-Palmer1941}{}%
Palmer, R. R. 1941. \emph{Twelve Who Ruled}. Princeton, NJ: Princeton
University Press.

\leavevmode\hypertarget{ref-Sider2012}{}%
Sider, Theodore. 2011. \emph{Writing the Book of the World}. Oxford:
Oxford University Press.

\leavevmode\hypertarget{ref-Vayrynen2013}{}%
Väyrynen, Pekka. 2013. \emph{The Lewd, the Rude and the Nasty: A Study
of Thick Concepts in Ethics}. Oxford: Oxford University Press.

\leavevmode\hypertarget{ref-Weatherson2019}{}%
Weatherson, Brian. 2019. \emph{Normative Externalism}. Oxford: Oxford
University Press.

\leavevmode\hypertarget{ref-Williams1985}{}%
Williams, Bernard. 1985. \emph{Ethics and the Limits of Philosophy}.
London: Routledge.

\end{CSLReferences}

\end{document}
