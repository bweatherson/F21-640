% Options for packages loaded elsewhere
\PassOptionsToPackage{unicode}{hyperref}
\PassOptionsToPackage{hyphens}{url}
%
\documentclass[
]{article}
\usepackage{amsmath,amssymb}
\usepackage{lmodern}
\usepackage{iftex}
\ifPDFTeX
  \usepackage[T1]{fontenc}
  \usepackage[utf8]{inputenc}
  \usepackage{textcomp} % provide euro and other symbols
\else % if luatex or xetex
  \usepackage{unicode-math}
  \defaultfontfeatures{Scale=MatchLowercase}
  \defaultfontfeatures[\rmfamily]{Ligatures=TeX,Scale=1}
  \setmainfont[]{SF Pro}
\fi
% Use upquote if available, for straight quotes in verbatim environments
\IfFileExists{upquote.sty}{\usepackage{upquote}}{}
\IfFileExists{microtype.sty}{% use microtype if available
  \usepackage[]{microtype}
  \UseMicrotypeSet[protrusion]{basicmath} % disable protrusion for tt fonts
}{}
\makeatletter
\@ifundefined{KOMAClassName}{% if non-KOMA class
  \IfFileExists{parskip.sty}{%
    \usepackage{parskip}
  }{% else
    \setlength{\parindent}{0pt}
    \setlength{\parskip}{6pt plus 2pt minus 1pt}}
}{% if KOMA class
  \KOMAoptions{parskip=half}}
\makeatother
\usepackage{xcolor}
\IfFileExists{xurl.sty}{\usepackage{xurl}}{} % add URL line breaks if available
\IfFileExists{bookmark.sty}{\usepackage{bookmark}}{\usepackage{hyperref}}
\hypersetup{
  pdftitle={PHIL 640: Empathy},
  pdfauthor={Brian Weatherson},
  hidelinks,
  pdfcreator={LaTeX via pandoc}}
\urlstyle{same} % disable monospaced font for URLs
\usepackage[margin=1.3in]{geometry}
\usepackage{graphicx}
\makeatletter
\def\maxwidth{\ifdim\Gin@nat@width>\linewidth\linewidth\else\Gin@nat@width\fi}
\def\maxheight{\ifdim\Gin@nat@height>\textheight\textheight\else\Gin@nat@height\fi}
\makeatother
% Scale images if necessary, so that they will not overflow the page
% margins by default, and it is still possible to overwrite the defaults
% using explicit options in \includegraphics[width, height, ...]{}
\setkeys{Gin}{width=\maxwidth,height=\maxheight,keepaspectratio}
% Set default figure placement to htbp
\makeatletter
\def\fps@figure{htbp}
\makeatother
\setlength{\emergencystretch}{3em} % prevent overfull lines
\providecommand{\tightlist}{%
  \setlength{\itemsep}{0pt}\setlength{\parskip}{0pt}}
\setcounter{secnumdepth}{-\maxdimen} % remove section numbering
\linespread{1.18}
\ifLuaTeX
  \usepackage{selnolig}  % disable illegal ligatures
\fi

\title{PHIL 640: Empathy}
\author{Brian Weatherson}
\date{November 22, 2021}

\begin{document}
\maketitle

\hypertarget{what-is-empathy}{%
\subsection{What is Empathy}\label{what-is-empathy}}

Most weeks when we're doing one of these things focussed on a particular
virtue, we start with questions about what it is, and why it might or
might not be a virtue. We're not going to quite do this here, because it
seems common ground that there are a lot of different things that we
might mean by empathy, and it doesn't really matter which of them the
ordinary term refers to. Still, it's worth clearing a few things out
first.

First question, which kind of state is core to empathy?

\begin{enumerate}
\def\labelenumi{\arabic{enumi}.}
\tightlist
\item
  Evaluative
\item
  Cognitive
\item
  Affective
\end{enumerate}

By `evaluative', I mean valuing the interests and well-being of others.
That doesn't seem like nearly enough for empathy. It isn't a particular
virtue, but the basis of morality or something close to it. Let's set
that aside.

Most of the writers we're interested in don't think merely cognitive
empathy is enough to count, but this is a bit more interesting. By
cognitive empathy, I mean that when someone is in a particular (morally
significant) way, the cognitively empathic person knows that they are
that way. So if A is sad, B knows that A is sad (or something like
this). And hopefully, if B is evaluatively empathic, B has a reason to
do something about this.

The central notion of empathy we're interested in, and again this is
more stipulation than analysis, is the affective notion. When A feels a
certain way, the affectively empathic person B will also feel that way.

Cross-cutting this three-way distinction, we can also ask whether
empathy requires something like exact match, or just something (morally)
close enough. This isn't a binary distinction, `close enough' is
obviously gradational. But it's helpful to simplify and think about the
more and less extreme versions. We'll call these the exact and inexact
versions of the theory.

So on the exact version, if A is bored, then exact, affective, empathy
requires B to feel bored (in the same way). But inexact, affective
empathy requires B to feel something close to that. Maybe feeling upset
or angry that A is bored is enough. And inexact cognitive empathy might
require merely that B knows that A feels badly in some way or other.

If we were trying to do conceptual analysis of the ordinary concept
EMPATHY, we might wonder next about whether some conjunctive or
disjunctive combination of these is the right account. Maybe ordinary
empathy requires exact evaluative and cognitive empathy and inexact
affective empathy. But let's not go there, because we're really
interested in exploring the conceptual space, not in matching up that
space to ordinary words. And noting these six basic ideas (three kinds
of empathy crossed with exact vs inexact) seems like enough.

One thing I will note is that some of Bloom's complaints, perhaps all of
the important ones, are targeted at exact affective empathy. He doesn't
think it's part of virtue that one is bored when people around you are
bored, or depressed when people around you are depressed. And I'm
inclined to agree - that doesn't seem like it's much good to anyone.
(Though we'll come back to reasons to worry about this.) But there are
lots of other notions of empathy.

\hypertarget{virtues-and-ability}{%
\subsection{Virtues and Ability}\label{virtues-and-ability}}

There is something that it's perhaps worth pausing over here, though
it's a bit delicate and I don't know quite what to say about it. The
following three claims are in some amount of tension.

\begin{enumerate}
\def\labelenumi{\arabic{enumi}.}
\tightlist
\item
  The six notions of empathy we started with (or at least four or five
  of those six) are \textbf{virtues}.
\item
  Many of these six notions represent attributes that it will be
  impossible, on broadly medical grounds, for people with certain
  disabilities to have.
\item
  Having a disability never makes one less virtuous.
\end{enumerate}

The general structure of the tension here is something that worries me a
lot in other contexts. I'm sympathetic to the Murdoch view that a core
part of morality involves seeing things in the right way. There are a
couple of ability-related reasons that you might worry about this view.
One is that the language is bad - you don't want to use language that
draws an analogy between good people and sighted people as opposed to
blind people. And that's fair enough, though if that's all that's going
on we should complain about the metaphor and work on fixing the
language. But there's a second deeper complaint. Maybe the view doesn't
just draw an analogy between physical sight and virtue, but genuinely
makes physical sight an aspect of virtue. And that really seems like
we're slipping onto uncomfortable territory.

And something like the same thing happens with cognitive and affective
empathy. The ability to know what others are feeling, and to feel the
same way, is an ability. That's to say, sometimes people don't have that
ability because they are disabled. That's fine, not everyone can do the
same things, and we don't normally make that morally significant. Saying
that someone is a less virtuous person because they don't have this
ability (to the same extent) is worrying. But I don't really have a good
story here. It's just something to keep in mind going forward.

\hypertarget{empathy-and-equality}{%
\subsection{Empathy and Equality}\label{empathy-and-equality}}

So here a couple of reasons for being sceptical of empathy in ethics.

\begin{enumerate}
\def\labelenumi{\arabic{enumi}.}
\tightlist
\item
  People who use empathy in forming judgments about what to prioritise
  do not make the judgments a utilitarian would approve of.
\item
  The right judgments are those that a utilitarian would approve of.
\item
  So it's bad to use empathy in forming judgments about what to
  prioritise.
\end{enumerate}

So obviously most people won't like premise 2 in that. (Though not
everyone!) When Bloom (and people like him) talk about empathy leading
to ``innumeracy'', I sometimes worry about the implicit argument being
something like this one. Could there be something better than could be
meant?

There is, I think, a more compelling argument in the vicinity. I'm not
going to make this valid, but you can see how to fill in the gaps I
hope.

\begin{enumerate}
\def\labelenumi{\arabic{enumi}.}
\tightlist
\item
  It's impossible, certainly in practice and maybe in theory, to
  empathise equally with everyone.
\item
  In practice (and maybe in theory) we empathise more with those who are
  already relatively privileged.
\item
  If we use empathy in forming priority judgments, we'll prioritise
  those we empathise more with.
\item
  It's bad to not treat people equally when forming priority judgments,
  and it's especially bad to violate equality by favoring those who are
  already privileged.
\item
  So, it's bad to use empathy in forming priority judgments.
\end{enumerate}

The `in practice' part of premise 1 should be obvious. But I think it's
a little interesting to think about the `in principle' part. Empathy is
sharing the experiences of others. But I don't know how you share the
experiences of many people at a time. It would violate the unity of
consciousness. So there's a fairly deep sense in which being empathic is
to privilege the one over the many.

Premise 2 should be reasonably clear as well, at least for people like
us. We find it easiest to empathise with people like us. If nothing
else, there are fewer epistemic barriers. And we are, at least in a
global context, incredibly privileged. But I suspect there is more to it
than that. Empathy involves focussing on the feelings of a particular
person. And it's easier to single out a privileged person than an
under-privileged one. Tolstoy's quip about families isn't really true
about individuals - a lot of underprivileged individuals are
underprivileged in depressingly similar ways. But privileged people get
distinctive privileges. Jeff Bezos and Elon Musk have really different
lives (even if both of them share an obsession for space travel).

Premise 3 seems relatively trivial, and hopefully premise 4 is
plausible. You don't have to be a consequentialist to think that some
basic principle of equality is central to ethics.

Still, while the premises don't look as utilitarian as the first
argument, you might think the conclusion will still be fairly
utilitarian. Or, at least, that it will lead to something like
utilitarianism with side-constraints. Maybe that's ok, but it's worth
thinking ahead to where this leads before you sign up for the premises.

\hypertarget{good-acts-and-good-people}{%
\subsection{Good Acts and Good People}\label{good-acts-and-good-people}}

One conclusion you might take from Bailey's paper is this:

\begin{itemize}
\tightlist
\item
  There are some goods that only bad (or at least suboptimal) people can
  bring about.
\end{itemize}

She presents this as a challenge within and for virtue ethics. But you
might equally take it to be a problem for a certain kind of weak
consequentialism. (This is, fwiw, how I take cases with this structure.)

Here's what I mean by `weak consequentialism'. Consider the view that
says it's more or less analytic that the right action is the one that
produces the most good. This isn't to say the right action produces the
most welfare, or surplus of pleasure over pain, or anything like that.
It's just that rightness is the thing that's produced by good actions.
And virtuous people, the view continues, just are those who do right
actions. The weak consequentialist, as I'm understanding them, doesn't
commit to any order of explanation here. They are committed to the
identity between right actions and actions typically performed by
virtuous people, but not to any order of explanation. It might be, as
Kant is typically read as holding, that the rightness of actions is
analytically prior. Or it might be that the virtues of agents are prior,
as modern virtue theorists think. Or it might be that goodness of
outcomes is analytically prior, as `stronger' consequentialists think.
All they insist on is the tie between these notions.

Sometimes this kind of view can seem like it is hard to argue against.
Perhaps you can argue against it by looking at various kinds of
dilemmas, and how they play out on different views. If you're
interested, the literature on this goes under the name
`consequentializing'. The person I'm calling a weak consequentialist
says all views can be `consequentialized'. And while not everyone
agrees, a lot of people think this works, at least as a verbal trick.

Anyway, Bailey's examples give us a way to directly argue against such a
view. Perhaps empathising with a bad person is something that (a)
produces goods, and (b) is something only a (somewhat) bad person can
do.

It's really crucial here that the empathy be partially
\textbf{constitutive} of the good, not merely part of the typical causal
background of the good. It's not a surprise that bad people might find
it easier to produce certain goods, or would be more likely to produce
certain goods. Here's one way that can happen. Imagine that X is bad not
in that they value bad things, but in that they fail to value most good
things. In particular, X only values one particular kind of good thing.
(It won't matter what it is.) Now goods that involve promoting that one
particular thing are more likely to be realised if X is around than if a
good person is around. That's because the good person might think that
promoting this one good has intolerable opportunity costs. But the
monomaniacal X, who has only one interest, will definitely promote it.
And we said the interest is something that's actually good. So there is
a good they are more likely to bring about. I don't think that's a
particularly interesting result. It's not surprising that when the right
thing to do involves some conflict, having a bad person around who
doesn't feel the force of the conflict could be useful. What would be
surprising is if the good person was genuinely incapable of producing
the good result. That's what Bailey's examples suggest.

You might wonder why this is a problem for the virtue theorist. After
all, the virtue theorist isn't a consequentialist. Couldn't they join
the deontologist in just saying that yeah, sometimes life is tragic, and
following the rules will lead to bad results? Well maybe some virtue
theorists could, but not the most common kind. Because the most common
virtue theorists say that what makes something a virtue is that it is
connected in the right kind of way to human flourishing. It would
require root-and-branch repair of the theory if we discovered that some
virtues were in fact inimical to human flourishing. And, again, that's
what the examples suggest.

\hypertarget{virtue-and-possibility}{%
\subsection{Virtue and Possibility}\label{virtue-and-possibility}}

What gets the puzzle going is the thought that good people simply won't
see bad actions as possible. There is something to this. I mentioned one
example in class last week, but maybe it's worth writing down.

A has fallen asleep on a park bench, with their phone in their lap. It's
right there to steal. I mean, it would be really easy to steal. Two
people walk by, B and C. B thinks about the phone, thinks about how easy
it would be to steal and sell it, and then thinks, no that would be
wrong. C looks at A and worries that the phone might get stolen because
there are bad people around, but concludes that moving the phone to make
it less visible would itself be a violation of A's space, so just hopes
for the best. Happily for all concerned, A soon wakes up, and goes back
to doomscrolling.

So question, which of these people are good? Well, don't worry about A -
they were just asleep. But how should we think about B and C. Bailey is
interested in a tradition, tracing to McDowell, that says only C is
truly virtuous. B is merely continent.

As an aside, I don't love the visuals of the truly bad person as
incontinent. It goes against the Austinian view that one can break the
rules with some amount of style.

\begin{quote}
I am very partial to ice cream, and a bombe is served divided into
segments corresponding one to one with the persons at High Table: I am
tempted to help myself to two segments and do so, thus succumbing to
temptation and even conceivably (but why necessarily?) going against my
principles. But do I lose control of myself? Do I raven, do I snatch the
morsels from the dish and wolf them down, impervious to the
consternation of my colleagues? Not a bit of it. We often succumb to
temptation with calm and even with finesse. (``A Plea for Excuses'',
198)
\end{quote}

Anyway, back to our main plot-line. What should we say about B and C?
McDowell says that C is the virtuous one. I suspect Murdoch should be
credited here as well - let's come back to that. (I don't mean to
criticise McDowell at all here; I think he is very up front about how he
is following Murdoch.) Bailey worries about the following objection. No
one is really like C. At least not always. Everyone is always a little
tempted by the benefits that theft has over honest toil. So the result
on this view will be that nobody is truly virtuous.

Armstrong used to say of certain theories he didn't like that they had
the benefits of theft over honest toil. I used to always think that if
that was the worst you could say about a theory, it might be time to
start checking where the metaphysical security guards were.

At this point you might think that yes, that's definitely a thing that
follows from the account of virtue. But is it anything more? That a
theory of virtue implies there are no perfectly virtuous people seems no
more a problem for it than it would be a problem if my theory of
unicorns implies the world is unicorn free. Do we have wonderful
independent evidence that the world is populated by moral saints? Are
they hiding under the sofa? I don't know, but I'm not moved by this at
all.

What I am moved by are two possibly related considerations. First, I
think there's a sense in which B is praiseworthy and C is not. B
overcame temptation. As Oscar Wilde almost quipped, that's the hardest
thing to overcome. Good for B! Second, you might think, in a broadly
Kantian spirit, that morality is all about intentional action. And C's
not considering the possibility of stealing the phone might be a
manifestation of a useful habit, but I don't think it's intentional. So
if morality is all about intentional action, there cannot be a sense in
which C is better than B.

So here's a big research question that I don't know the answer to. How
much does this kind of reasoning drive Murdoch's anti-Kantianism? That's
for another kind of seminar I'm afraid.

\hypertarget{silencing}{%
\subsection{Silencing}\label{silencing}}

What do McDowell, and Bailey, mean when they say that reasons to perform
bad acts are \emph{silenced}. They don't mean the kinds of thing people
usually mean these days in philosophy by silencing. This isn't anything
from the Hornsby-Langton school about how speech acts can't be
recognised as such. What's silenced here is not a person but a reason.

As Bailey notes, there are a bunch of different looking things that
McDowell might mean by this. But they are only different looking. They
have different meanings on different ways to translate McDowell into
more familiar Humean-speak. McDowell, however, is not a Humean. And
sometimes that means if you try to specify a theory in Humean language
and ask which bit he disagrees with, you won't get a satisfying answer.
Because what you need to say is that the translation misses something.

This is a philosophical problem - how do we understand a disagreement
between people when we struggle to articulate a thesis that one accepts
and the other rejects? But it's also a practical problem for me, because
I'm enough of a Humean that at some level what I'm doing when I'm
reading non-Humean works is translating them into my language, and
seeing what I disagree with. And that might be inappropriate here. But
that said, I think I have some sense of what silencing means.

\begin{quote}
Silencing is when an option is taken off the table because it is morally
unacceptable. We don't compute the expected costs and benefits, goods
and bads, rights and wrongs, of the action. We just don't consider it as
one of the options.
\end{quote}

I'm going to go with that interpretation, although you really should
note that it's my best guess, and this is not really my field. One
interesting consequence of this interpretation (or perhaps reason to
think it isn't right) is that it isn't only reasons \emph{for} doing
immoral actions that are silenced. It's also reasons \emph{against}
doing them. Imagine I'm thinking of stealing a car. And there is a real
downside to this, I might get caught. That's a reason for me not to
steal the car. On my way of understanding the view, that reason is
silenced for the virtuous agent. They just don't think about stealing
the car.

For what it's worth, I sort of think this picture is built into the
criminal law. It's true that the criminal law has punishments that are
meant to disincentivise. But I think the picture the law has is that
these are meant to be redundant for the good citizen. The good citizen
doesn't weigh up the expected costs and benefits of stealing a car, and
the role of retrospective punishment is not merely to alter that
retrospective calculation. Rather, the good person hears the
pronouncement \emph{Thou shalt not steal cars} and doesn't steal cars.
Of course, not all people are good people and the punishments have to be
harsh enough to disincentivise the bad people. But that's a secondary
role for the criminal law. End of digression on crime I guess.

But it's going to be important for what follows that we be careful about
what exactly is silenced. Think about Bailey's example of the pears. On
this picture, the virtuous person won't think of stealing the pears as
even an option, so they won't think about reasons for or against
stealing them. So far so good. But that doesn't mean they can't think
about the pears. If you ask them ``If, purely hypothetically, one could
legitimately acquire these pears, they'd taste really good wouldn't
they?'', they would surely answer yes. Silencing doesn't mean you can't
ask hypotheticals. It's just that if you ask them ``If, purely
hypothetically, I just stole these pears and ate them, I'd really enjoy
them wouldn't I?'', they'd faint or something. (I'm not sure what the
intended causal mechanism is by which these reasons are silenced.)

The point here is that the virtuous person still knows that pears are
tasty, and knows that is a reason to eat pears, and knows that this is a
reason to possess pears. They just don't see it as a reason to
\emph{steal} pears. And this means that there are limits to the limits
to their empathy. Although Bailey doesn't make a big point of this
worry, I think she's sensitive to it, and that's why she constructs some
of her examples the way that she does.

\hypertarget{understanding}{%
\subsection{Understanding}\label{understanding}}

What's the good thing that only bad people allegedly can do? It's to
have a certain kind of \textbf{understanding}. Linking back to what we
discussed at the start, there are two aspects I think to this kind of
understanding.

\begin{enumerate}
\def\labelenumi{\arabic{enumi}.}
\tightlist
\item
  One is to do with reasons. It's to appreciate the reasons that someone
  has, and to appreciate them as reasons. So it's not just to know that
  pears are tasty, but to appreciate that the tastiness of the pears is
  a reason for someone to act.
\item
  The other is to do with phenomenology. It's to know what it's like to
  be in the position that the person being understood is in. And that
  probably (maybe) requires having that experience, or at least
  something very much like it.
\end{enumerate}

You might just about think that these two are closely connected. It's
much easier to appreciate something as a reason if you know what it
feels like to get it and to lose it. But we have to be a little careful
here. The connection I just stated is \textbf{causal}, and what we're
really interested in here are \textbf{constitutive} connections.

As noted earlier, it isn't a surprise that some bad people are more
useful in some situations than good people. What would be surprising is
if there is something a good person cannot do, just in virtue of their
being good people. And here the question is whether they can provide the
sort of empathic, humane understanding that friends need and perhaps
deserve.

There are a couple of different cases that Bailey talks about, and which
I had different reactions to: the pears and the schadenfreude. I wasn't
really convinced by the pears example. True, the virtuous person (on
McDowell's view of virtue) can't empathise with the desire to
\emph{steal} the pears. But they can empathise with the desire to
\emph{possess} the pears. They know that pears are tasty, they even know
the way in which they are tasty, and they can appreciate that as a
reason for wanting to possess them. It's a very strange kind of virtue
that says that if in fact the only way to possess the pears is
illegitimate, the virtuous person cannot even imagine legitimate
possession, and the benefits that come with it. Imagination isn't bound
by factual niceties like this, and imagination should be enough. So I
don't really think that's a clear example. Maybe the virtuous person
will find it harder to empathise, because they have to imagine the facts
being a little different so the possession is legitimate, but that's not
I think a big deal.

The schadenfreude example is much more interesting. Imagine A is
virtuous, B is A's vicious friend, and C is B's enemy. B feels bad
because something bad almost, but did not, happen to C. And they want A
to empathise. And here I can see why it would be harder for the virtuous
person. The first solution to the puzzle Bailey mentions, imaging that A
could have an offline mind that can see what it's like to desire
schadenfreude and an online part that doesn't, doesn't feel plausible.
At least, I can't picture being so invested in the picture of virtue
that makes this a problem and also feeling that this offline/online
picture makes sense.

But maybe we should ask how big a deal it is if you need some bad people
around for this. What's the cost of saying that only bad people can help
you if you need someone to empathise over the failures of your bad
schemes? Put that way, it sounds like something we should maybe expect.
So let's leave it with that point - what's the downside of just having
this outcome?

\end{document}
