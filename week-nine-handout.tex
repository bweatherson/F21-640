% Options for packages loaded elsewhere
\PassOptionsToPackage{unicode}{hyperref}
\PassOptionsToPackage{hyphens}{url}
%
\documentclass[
]{article}
\usepackage{amsmath,amssymb}
\usepackage{lmodern}
\usepackage{iftex}
\ifPDFTeX
  \usepackage[T1]{fontenc}
  \usepackage[utf8]{inputenc}
  \usepackage{textcomp} % provide euro and other symbols
\else % if luatex or xetex
  \usepackage{unicode-math}
  \defaultfontfeatures{Scale=MatchLowercase}
  \defaultfontfeatures[\rmfamily]{Ligatures=TeX,Scale=1}
  \setmainfont[]{SF Pro}
\fi
% Use upquote if available, for straight quotes in verbatim environments
\IfFileExists{upquote.sty}{\usepackage{upquote}}{}
\IfFileExists{microtype.sty}{% use microtype if available
  \usepackage[]{microtype}
  \UseMicrotypeSet[protrusion]{basicmath} % disable protrusion for tt fonts
}{}
\makeatletter
\@ifundefined{KOMAClassName}{% if non-KOMA class
  \IfFileExists{parskip.sty}{%
    \usepackage{parskip}
  }{% else
    \setlength{\parindent}{0pt}
    \setlength{\parskip}{6pt plus 2pt minus 1pt}}
}{% if KOMA class
  \KOMAoptions{parskip=half}}
\makeatother
\usepackage{xcolor}
\IfFileExists{xurl.sty}{\usepackage{xurl}}{} % add URL line breaks if available
\IfFileExists{bookmark.sty}{\usepackage{bookmark}}{\usepackage{hyperref}}
\hypersetup{
  pdftitle={PHIL 640: Hypocrisy},
  pdfauthor={Brian Weatherson},
  hidelinks,
  pdfcreator={LaTeX via pandoc}}
\urlstyle{same} % disable monospaced font for URLs
\usepackage[margin=1.3in]{geometry}
\usepackage{graphicx}
\makeatletter
\def\maxwidth{\ifdim\Gin@nat@width>\linewidth\linewidth\else\Gin@nat@width\fi}
\def\maxheight{\ifdim\Gin@nat@height>\textheight\textheight\else\Gin@nat@height\fi}
\makeatother
% Scale images if necessary, so that they will not overflow the page
% margins by default, and it is still possible to overwrite the defaults
% using explicit options in \includegraphics[width, height, ...]{}
\setkeys{Gin}{width=\maxwidth,height=\maxheight,keepaspectratio}
% Set default figure placement to htbp
\makeatletter
\def\fps@figure{htbp}
\makeatother
\setlength{\emergencystretch}{3em} % prevent overfull lines
\providecommand{\tightlist}{%
  \setlength{\itemsep}{0pt}\setlength{\parskip}{0pt}}
\setcounter{secnumdepth}{-\maxdimen} % remove section numbering
\linespread{1.18}
\ifLuaTeX
  \usepackage{selnolig}  % disable illegal ligatures
\fi

\title{PHIL 640: Hypocrisy}
\author{Brian Weatherson}
\date{November 8, 2021}

\begin{document}
\maketitle

\hypertarget{notions-of-hypocrisy}{%
\subsection{Notions of hypocrisy}\label{notions-of-hypocrisy}}

First big question: What is hypocrisy? Dover lists a number of distinct
things that go under the label, not all of which naturally fit together.

\begin{enumerate}
\def\labelenumi{\arabic{enumi}.}
\tightlist
\item
  Violating the \emph{Do unto others} rule vis a vis forgiveness.
\item
  Not living up to our own ideals.
\item
  Not examining our own behaviour the way we examine others.
\item
  Engaging in self-flattering sanctimoniousness.
\item
  Criticising others for things that we do.
\end{enumerate}

The first, third and fifth all involve comparisons between what one
person does and how they relate to others. That, at least, seems to be
at the heart of the philosophical literature. (And of Dover's paper.) So
let's ignore the second and fourth. Hypocrisy is something about
treating self and others differently. But what is the different
treatment? Here are a bunch of questions.

\begin{itemize}
\tightlist
\item
  Must it involve a perfect parallel?
\end{itemize}

For some writers, it does. It involves, for instance, blaming others for
misdeeds that one does not blame oneself for. Dover's core notion does
not. It involves criticising others for an action one does. But some of
her characters do criticise themselves for that action. Some don't, and
she has less to say than you might like about whether that matters.

\begin{itemize}
\tightlist
\item
  Must it involve action towards others?
\end{itemize}

For some writers, hypocrisy involves a public action, like criticising
or blaming. For others, it suffices to have a private attitude towards
actions by others that one does not have towards oneself.

\begin{itemize}
\tightlist
\item
  Is it no longer hypocrisy if you apologise/self-blame/atone for the
  wrong?
\end{itemize}

This may follow from the answer to the parallel question. But there is
an interesting issue that charges of hypocrisy feel less forced against
people who acknowledge their own wrongdoing.

\begin{itemize}
\tightlist
\item
  Is the relevant public action criticism or blame?
\end{itemize}

This feels like a big deal, and a lot of the noise here comes from
running them together. Any plausible view of blame will have it be
something stronger than criticism.

\begin{itemize}
\tightlist
\item
  Empiricial question: Does the person who sincerely disapproves of
  their own actions, and criticises themselves as harshly as they
  criticise others, get labeled a hypocrite in normal discourse? I'm
  really not sure.
\end{itemize}

\hypertarget{law-and-standing}{%
\subsection{Law and Standing}\label{law-and-standing}}

One running theme in the readings is the idea that we should take the
notion of a moral \textbf{law} seriously. And just as in secular law
only certain people have \textbf{standing} to bring legal cases, maybe
in moral law the same applies. And one way to lose standing is to be
guilty of the same sins one accuses others of.

This notion of standing feels very suspicious. For one thing, the moral
law/secular law analogy can, I think, be taken too seriously. For
another, the whole field of standing is one of the worst areas of
secular law. So I don't know that we should be going out of our way to
bring it into moral law.

There are also a bunch of things different people in the readings say
about standing that feel like they are talking at cross purposes.

\begin{itemize}
\tightlist
\item
  Fritz and Walker think that it is something that follows from equality
  of persons, but which can be defeated by hypocrisy. This seems wrong
  twice over. For one thing, it suggests that only persons have standing
  to blame. And that's false. Institutions can blame, pets can (and do!)
  blame, maybe even algorithms can blame. For another, they end up with
  a kind of `finkish' standing, where a person allegedly has standing to
  do something, but if they did it, they would lose that standing. This
  seems like a useless theoretical device.
\item
  Lippert-Rasmussen has a more interesting take. He thinks standing is
  just to do with public blame. And it's not really the standing to
  perform the underlying act that matters, but standing to demand a
  certain kind of response. (We could have some discussion here about
  whether demanding that response is constitutive of the underlying
  speech act.) And it is a bit plausible that something like hypocrisy
  could remove your ability to make this kind of demand on others. But
  note that even without this standing, you'd still (for all we've said
  so far) be able to silently seethe about others doing things you do.
\end{itemize}

A better way of thinking about hypocrisy comes from thinking about blame
(and perhaps other kinds of criticism) as forms of \textbf{altruistic
punishment}. At its worst, this can seem like vigilante justice. At its
best, it is a low cost way of enforcing social norms.

One problem with altruistic punishment, one we've seen a lot over the
years, is that if everyone punishes the same person for an offense, the
punishment can be excessive. A good system of altruistic punishment
should put some limits to that. Some of these are physical limits. It
could be that there are some interesting ties to social media here - one
thing that goes wrong is that heuristics for when it is appropriate to
engage in social punishment fail when we have such close `contact' with
everyone. But anti-hypocrisy rules could be part of the social rules
limiting informal punishment - it's a way of keeping punishment
proportionate. The more people who have done what you did, the less
punishment you can possibly get. That seems maybe fair and a reasonable
way of limiting punishment.

\hypertarget{wallace-argument}{%
\subsection{Wallace Argument}\label{wallace-argument}}

\begin{description}
\tightlist
\item[INTEREST]
Criticism is something that we all have an interest in avoiding.
\item[TREATMENT]
When I criticize you for X'ing although I X myself, I treat your
interest in avoiding being criticized as less important than my own.
\item[EQUALITY]
When I treat your interests as less important than my own, I offend
against the equality of persons.
\end{description}

Dover's primary response to this turns on the importance of moral
communication in moral education.

\begin{itemize}
\tightlist
\item
  We all have an interest in moral education.
\item
  Given human limitations, this interest is best served by our being
  willing to listen to criticism from others, and to give it out.
\item
  Of course there are more or less jerkish ways to criticise others, but
  that doesn't mean the criticism itself is bad
\end{itemize}

Dover's view here tracks the Mercier and Sperber view that the point of
reason from an evolutionary perspective is to win arguments - we're
evolved to be good at spotting mistakes that others are making.

\hypertarget{dover-on-criticism-and-blame}{%
\subsection{Dover on Criticism and
Blame}\label{dover-on-criticism-and-blame}}

Here's one possible position. I'm not quite sure it is Dover's, but it
seems like what she's getting at on 395ff.

\begin{itemize}
\tightlist
\item
  The sanction/informal punishment view of blame is plausible enough
\item
  And like all punishment regimes, there should be rules in place for
  keeping it proportionate, so in principle we could justify
  anti-hypocrisy rules as part of what keeps social punishment
  proportionate.
\item
  But \textbf{criticism is not blame}, and is not punishment.
\item
  At least, criticism of the `exhortation and advice' form isn't, even
  if it is advice to not do the thing that you just did.
\item
  Rather, this kind of criticism is a crucial aspect of moral
  communiation,
\item
  And a rule against communicating about X if you've acted wrongly with
  respect to X has no particular motivation or benefit.
\end{itemize}

To my eyes the big question here is whether it makes sense to separate
out blame and criticism in this way. Are forms of criticism that do not
come with any kind of social sanction - especially outside the context
of a pre-existing relationship?

Conversely, is being the target of silent seething as bad as Dover makes
it out to be? Is that what we should identify blame with? It feels to me
blame is much more public than that.

\newpage

\hypertarget{benefits-of-moral-conversation}{%
\subsection{Benefits of Moral
Conversation}\label{benefits-of-moral-conversation}}

Dover lists four benefits of moral conversation. She only numbers three,
but I think the unnumbered first one is in some ways the most important.

\begin{itemize}
\tightlist
\item
  Educating onlookers. I think this gets overlooked a lot. Sometimes a
  technique, such as shaming the wrongdoer, might be rather ineffective
  at changing the behavior of the purported target, but be very
  effective at changing the behavior of onlookers. This is something
  that I think gets overlooked in a lot of discussions of moral
  communication.
\item
  Conversation makes us give reasons. This is something that gets left
  out of a lot of debates on disagreement in epistemology. The benefit
  of conversing about X is not just that we learn A thinks X is wrong,
  but we hear A's reasons that they think X is wrong. And hopefully at
  least sometimes they will be reasons that we can follow as well. That
  is, sometimes we will see that there are reasons for not doing what we
  do that we simply hadn't appreciated.
\item
  Conversation lets us see how others see us.
\item
  They are a source of ``affective and cognitive energy''. Preaching to
  the choir has value!
\end{itemize}

But then Dover makes some strong claims about the role of personal
criticism in promoting these values.

\begin{quote}
None of these three roles can be played in quite the same way by
bloodless abstract debates, by joint deliberation that aims to produce
consensus at the end of the day, or by personal exchanges in which anger
and frustration are suppressed in an attempt to keep the discussion
purely constructive.
\end{quote}

Are we sure? What's the argument for this?! If they can't be played in
the same way, can they be played in as good a way, or a better way?

\begin{quote}
Nor can they be played very well by interpersonal moral criticism, so
long as that is understood as a peremptory form of moral address aimed
exclusively at holding presumptive wrongdoers accountable and upholding
uncontroversial moral norms.
\end{quote}

And here I think there is at least one misstep. This is at best
misleading about where the blame/criticism boundary goes. ``You
shouldn't have done that'' is not ``peremptory'' but nor is it blame.
Dover is assuming that ``You shouldn't have done that'' is the bad form
of criticism that, according to Wallace, we all have an interest in
avoiding. And maybe that's what Wallace thinks, I'm not sure. But it
seems like we could keep 90+\% of Wallace's view while saying that
non-blaming criticism like this falls under the category of ``moral
exhortation and advice''.

\hypertarget{five-cases}{%
\subsection{Five Cases}\label{five-cases}}

I want to go over the five cases that Dover lists, so it's helpful to
have a list of them.

\begin{enumerate}
\def\labelenumi{\arabic{enumi}.}
\tightlist
\item
  Weak willed person who criticises others for doing what he tries and
  fails to prevent himself doing,
\item
  Insincere person who gets instrumental gain from criticising something
  he doesn't disapprove of, and in fact does.
\item
  A person with double standards who implicitly or explicitly holds
  others to higher standards, e.g., because of gender.
\item
  A person who is overly critical of others, while complacent about
  their own actions. This seems like the normal case that people are
  interested in.
\item
  A morally resigned person who does something that he thinks is wrong,
  but isn't even trying to fix.
\end{enumerate}

Dover disagrees with Bell's claim that people in case 3 will usually be
`exception seeking'. I'm more sympathetic to Bell here. I think it's
really common that people in fact have double standards, but they
present themselves (to themselves as well as to others) as having a
principled exception to a general principle that just so happens to help
in their own case. That seems to me much more common than the explicit
double standard that Dover discusses.

Case 5 is fairly interesting because it really doesn't have any kind of
double standards. The person doing the criticism wishes that they had
been criticised in the same kind of way. So the target Dover is
interested in here is compatible with a really strong kind of
recognition of moral equality.

\hypertarget{dovers-positive-view}{%
\subsection{Dover's Positive View}\label{dovers-positive-view}}

\begin{itemize}
\tightlist
\item
  In any case where there is hypocrisy, there is some other explanation
  of what's gone wrong.
\item
  We wouldn't prefer that the hypocrisy is fixed; we'd prefer the
  underlying problem is fixed.
\item
  This comes up especially clearly at the bottom of 414. It's better to
  have one mistake rather than two, and changing to being coherent,
  avoiding the hypocrisy, might mean we end up with two mistakes.
\end{itemize}

There is a reliance on counting arguments here that seems to apply
equally to any kind of coherence constraint.

\begin{itemize}
\tightlist
\item
  Any time someone says it is good to be coherent, you can reply that
  there are already mistakes in anyone incoherent, and that some ways of
  restoring coherence are bad, in so far as they would produce two
  errors rather than one.
\item
  Maybe you're happy saying ``Yeah, down with coherence constraints.''
\item
  But that's a really strong conclusion, and one we should be sceptical
  about.
\end{itemize}

\hypertarget{two-kinds-of-permissivism}{%
\subsection{Two kinds of permissivism}\label{two-kinds-of-permissivism}}

There are multiple things that we can morally permissively do. Everyone
except the most extreme consequentialist thinks this.

But here's a somewhat stronger kind of permissivism. There are multiple
moral views we can have on the world. In particular, we can draw the
line between sub-optimal and impermissible at points of our own
choosing. Whether to criticise someone for doing something clearly
sub-optimal might be up to us, and in particular up to whether we judge
it not just sub-optimal, but wrong. This is at least a kind of
relativism, though a somewhat idiosyncratic kind.

If it's true, you might think there is a meta-norm - draw the lines
consistently. Don't draw them differently for different sexes, or
genders, or races, or self-other. What goes wrong if you violate that?
Not that either choice is wrong, if they are in permissive range. We
can't explain the wrongness here by counting up the misdemeanours.
Rather, the problem is a straight up violation of equality. I'm not sure
how Dover could handle this. I think she's assuming this isn't the right
way to think about morality. But I'm not sure anti-hypocrisy norms make
sense without this kind of picture.

\hypertarget{hypocrisy-and-moral-equality}{%
\subsection{Hypocrisy and Moral
Equality}\label{hypocrisy-and-moral-equality}}

One big theme of the readings has been the following principle.

\begin{itemize}
\tightlist
\item
  What's wrong with hypocrisy is that the hypocrite doesn't respect the
  equality of persons.
\end{itemize}

Now most of the authors reject this line of reasoning, either rejecting
that there is something wrong with hypocrisy or that this is the
explanation for it. Lipper-Rasmussen argues that it has to fail for the
following reason.

\begin{description}
\tightlist
\item[Relevantly Similar Equality Claim]
Hypocrites and hypercrites are relevantly similar with regards to the
denial of moral equality of persons.
\item[Standing Asymmetry Claim]
Hypocrites lack standing to blame but hyper-crites do not lack standing
to blame.
\item[Anti-Equality Conclusion]
So what explains hypocrites' lack of standing cannot be their denial of
moral equality of persons.
\end{description}

So one thing is for us to discuss this argument, and whether it works.
And whether, as Tierney suggests, it also undermines Lippert-Rasmussen's
positive proposal.

But more generally, I think we should be very suspicious of the claim
that equality requires us to blame everyone alike. We simply don't do
anything like this in real life. If someone in this class is rude to
someone else, I'll blame them for it, without equally blaming every
instance of rudeness everywhere in the world. To get from equality to an
anti-hypocrisy norm, we need something about how this is ok, but
hypocrisy is not. And none of the attempts to bridge this gap seemed
remotely successful to me.

There is also a question about what kind of equality violation is in
question. Here are three ways of spelling out moral equality.

\begin{description}
\tightlist
\item[Equality of Beliefs]
One should believe that people generally are equal, and of arbitrary
people that they are equal.
\item[Equality of Value]
One should value different people equally.
\item[Equality of Treatment]
One should treat different people equally.
\end{description}

And there's a general challenge for any kind of argument from equality
to anti-hypocrisy.

\begin{itemize}
\tightlist
\item
  If you start with one of the first two, it's hard to get from
  hypocrisy to a violation. You can't tell from my hypocritical
  behaviour that I violate either of the first two. I might genuinely
  not have noticed that I've done anything wrong. I might be weak-willed
  about blaming/criticising myself. I might believe that people are
  equal and just not act on my beliefs, etc.
\item
  If you start with the third, you need to qualify it in a way that,
  e.g., treating one's own children differently to how one treats others
  is consistent with the equality principle. And once you do that, it is
  going to be hard to make the argument work - the exceptions you add
  will be big enough to drive the hypocritical treatment through.
\end{itemize}

\end{document}
